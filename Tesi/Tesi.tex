\documentclass[
	a4paper,
	twoside,
	12pt
]{book}

\usepackage{adjustbox}
\usepackage{amsmath}
\usepackage{amsthm}
\usepackage[italian]{babel}
\usepackage{bm} 
\usepackage{caption}
\usepackage[style=italian]{csquotes}
\usepackage{diagbox}
\usepackage{enumitem}
\usepackage{fancyhdr}
\usepackage[T1]{fontenc}
\usepackage{frontespizio}
\usepackage[top=3cm, bottom=3cm, left=3cm, right=3cm]{geometry}
\usepackage{graphicx}
\usepackage[colorlinks=true, linkcolor=blue, urlcolor=blue, citecolor=blue]{hyperref}
\usepackage{parskip}
\usepackage{placeins}
\usepackage{setspace}
\usepackage{siunitx} 
\usepackage{todonotes}
\usepackage{xcolor}

\newtheoremstyle{StileEsempio}
  {1.5em}   
  {1.5em} 
  {\normalfont}
  {}         
  {\bfseries}
  {.}         
  {.5em}     
  {} 
\theoremstyle{StileEsempio}
\newtheorem{esempio}{Esempio}

\graphicspath{{./Immagini}}
\onehalfspacing

\pagestyle{fancy}
\fancyhf{}
\fancyfoot[LE,RO]{\thepage}
\renewcommand{\headrulewidth}{0pt}
\fancypagestyle{plain}{%
  \fancyhf{}%
  \fancyfoot[LE,RO]{\thepage}%
  \renewcommand{\headrulewidth}{0pt}%
}

\setlength{\parindent}{0pt}
\setlength{\parskip}{1em}
\raggedbottom

\begin{document}
%Inizio Introduzione Tesi
\frontmatter
%Frontespizio
\begin{frontespizio}
	\Margini{3cm}{3cm}{3cm}{3cm}
	\Universita{Bergamo}
	\Logo[43.332mm]{./Immagini/Frontespizio/logo_unibg.pdf}
	\Divisione{Scuola di Ingegneria}
	\Corso[Laurea Triennale]{Ingegneria Informatica\\Classe n. L-8 Ingegneria dell’Informazione (D.M. 270/04)}
	\Titolo{Introduzione al calcolo parallelo in MATLAB\textsuperscript{\textregistered}}
	\Candidato[1086063]{Thomas Fabbris}
	\Relatore{Chiar.mo Prof.\ Fabio Previdi}
	\Annoaccademico{2024--2025}

	\begin{Preambolo*}
		\usepackage[italian]{babel}
		\usepackage[T1]{fontenc}
		\usepackage[utf8]{inputenc}
		\usepackage{microtype}
		\usepackage{lmodern}
		\usepackage{bm}

		\renewcommand{\frontinstitutionfont}{\fontsize{14}{17}\bfseries\scshape}
		\renewcommand{\fronttitlefont}{\fontsize{17}{21}\bfseries\scshape}
		\renewcommand{\frontfootfont}{\fontsize{12}{14}\bfseries\scshape}
	\end{Preambolo*}
\end{frontespizio}

% Indice
\tableofcontents

\mainmatter
% Introduzione della tesi
\chapter*{Introduzione}
\addcontentsline{toc}{chapter}{Introduzione}
%!TeX root = ../../Tesi.tex
L'obiettivo di questo capitolo \`e esibire una definizione puntuale di calcolo parallelo, un termine impiegato nel mondo
dell'HPC (\textit{High Performance Computing}) per riferirsi all’uso simultaneo di molteplici risorse di calcolo, consentendo la risoluzione di problemi a
elevata intensit\'a computazionale in tempi ragionevolmente brevi.

In seguito, investigheremo le cause che portarono alla nascita del parallelismo e descriveremo le principali difficolt\`a incontrate dai programmatori di
applicazioni durante l'implementazione di programmi a esecuzione parallela.
% Capitolo 1 (Calcolo parallelo: sfida o opportunità?)
\chapter{Calcolo parallelo: sfida o opportunit\'a?}
\label{cap1}
% Introduzione Capitolo 1
L'obiettivo di questo capitolo \'e dare una definizione precisa di calcolo parallelo, un termine spesso impiegato nel mondo
del supercalcolo per riferirsi all’uso simultaneo di molteplici risorse di calcolo per la risoluzione di problemi ad elevata intensità computazionale in tempi ragionevolmente brevi.

In seguito, investigheremo le cause che hanno portato alla nascita del parallelismo e a descrivere le principali difficolt\'a incontrate dai programmatori durante l'implementazione di programmi ad esecuzione parallela.
% Paragrafo 1.1: Introduzione al calcolo parallelo
\section{Introduzione al calcolo parallelo}
\label{par1.1}
L'idea alla base del calcolo parallelo \'e che gli utenti di un qualsiasi sistema di elaborazione possono avere a disposizione tanti processori
quanti ne desiderano, per poi interconnetterli a formare un sistema
multiprocessore, le cui prestazioni sono, con buona approssimazione,
proporzionali al numero di processori montati.

La sostituzione di un singolo processore caratterizzato da un'elevata
capacit\'a di calcolo, tipicamente presente nelle architetture di sistemi
\textit{mainframe}, con un insieme di processori pi\'u efficienti
dal punto di vista energetico permette di raggiungere migliori prestazioni
per unit\'a di energia, a condizione che i programmi eseguiti siano
appositamente progettati per sfruttare pienamente la potenza di calcolo di ogni
singolo processore; approfondiremo questi aspetti nei paragrafi \ref{par1.2} e \ref{par1.3}.

Una tendenza introdotta da IBM nel 2001 nell'ambito della progettazione di sistemi paralleli \cite{tendler2001power4} è il raggruppamento
di diverse unit\'a di calcolo all'interno di un singolo circuito integrato; in questo contesto, i processori montati su un singolo \textit{chip} di silicio vengono chiamati \textit{core}.
Il microprocessore \textit{multicore} risultante appare al sistema operativo in esecuzione sull'elaboratore come l'insieme di $N$ processori, ognuno dei quali dotato di un set di registri e di una memoria \textit{cache} dedicati; solitamente si tratta di sistemi multiprocessore a memoria condivisa, in cui i \textit{core} condividono lo stesso spazio di indirizzamento fisico.\newline
Il funzionamento di questa categoria di sistemi multiprocessore si basa sul parallelismo a livello di attività (o a livello di processo): più
processori sono impiegati per svolgere diverse attivit\'a simultaneamente e ciascuna attivit\'a corrisponde ad applicazioni a singolo
\textit{thread}. In generale, ogni \textit{thread} esegue un'operazione ben definita e \textit{thread} differenti possono agire sugli stessi
dati o su insiemi di dati diversi.

D'altro canto, tutte le applicazioni che richiedono un utilizzo intensivo di risorse computazionali, oggi diffuse non pi\'u esclusivamente in ambito
scientifico ma anche in settori come quello dei motori di ricerca o dell'\textit{hosting} di siti Web, possono essere eseguite solo su \textit{cluster} di elaboratori, una tipologia di sistemi multiprocessore che si differenzia dai microprocessori \textit{multicore} per il fatto di essere costituita da un insieme di calcolatori completi, chiamati nodi, collegati tra loro per mezzo di una rete LAN (\textit{Local Area Network}).\newline
In ogni caso, il funzionamento di un sistema di elaborazione parallelo si basa sull'uso congiunto di processori diversi.\newline
Per sfruttare al meglio le potenzialit\'a offerte dai \textit{cluster} di elaboratori, i programmatori di applicazioni devono sviluppare programmi a esecuzione parallela efficienti e scalabili a seconda del numero di processori a disposizione durante l'esecuzione; risulta necessario applicare il parallelismo a livello di dati, un approccio che prevede la distribuzione di un insieme di dati di partenza sulle CPU del \textit{cluster} per poi eseguire la medesima operazione, ma su un insieme di dati diverso, su ogni processore.

Una tipica operazione parallelizzabile a livello di dati è la somma vettoriale poichè ogni componente del vettore risultante viene ottenuta
sommando le corrispondenti componenti omologhe dei due vettori di partenza, in modo indipendente dalle altre; possiamo sin da subito intuire che una condizione necessaria per la parallelizzazione di un qualsiasi algoritmo è l'indipendenza tra le operazioni eseguite in un certo passo.
Ad esempio, supponiamo di dover sommare due vettori di numeri reali di dimensione $N$ avvalendoci di un sistema \textit{dual-core}, ossia di un sistema parallelo dotato di un microprocessore con
due \text{core} distinti. Sarebbe conveniente avviare un thread separato su ogni \textit{core} specializzato nella somma di due componenti omologhe dei vettori operandi; attraverso una opulata distribuzione dei dati, otterremmo che il \textit{thread} sul primo \textit{core} somma le componenti omologhe dei vettori di partenza
da $1$ a $\left\lfloor\frac{N}{2}\right\rfloor$, mentre il secondo \textit{core} si occuperebbe della somma delle componenti da $\left\lceil\frac{N}{2}\right\rceil$ a $N$.

In realtà, la rigida classificazione proposta tra parallelismo a livello di attivit\'a e parallelismo a livello di dati non ritrova un diretto
riscontro nella realt\'a, in quanto sono stati sviluppati programmi che sfruttano entrambi gli approcci al fine di massimizzare le prestazioni.

Cogliamo l'occasione per precisare la terminologia, in parte gi\'a utilizzata, impiegata per descrivere la componente hardware e la componente software di un sistema di elaborazione: l'hardware, descrivendo con questo termine il processore del sistema, pu\'o essere classificato in seriale, come nel caso di un processore \textit{single core}, o parallelo, come nel caso di un processore \textit{multicore}, mentre il software viene detto sequenziale o concorrente, a seconda della presenza o meno di diversi processi cooperanti la cui esecuzione \'e influenzata dagli altri processi presenti nel sistema.\newline
Naturalmente, un programma concorrente pu\'o essere eseguito sia su hardware seriale che su hardware parallelo.\newline
Infine, con il termine programma ad esecuzione parallela, o pi\'u semplicemente software parallelo, indicheremo un programma (sequenziale o concorrente) eseguito su hardware parallelo.
% Paragrafo 1.2: Le cause a supporto del parallelismo
\section{Le cause a supporto del parallelismo}
\label{par1.2}
L'attenzione riservata all'elaborazione parallela da parte della comunit\'a
scientifica risale al 1957, anno in cui la
Compagnie des Machines Bull (l'odierna Bull SAS) ha annunciato Gamma 60, un computer \textit{mainframe}
equipaggiato con la prima architettura della storia in grado di offrire un supporto diretto
al parallelismo, mentre l'anno successivo, i ricercatori IBM John
Cocke e Daniel Slotnick hanno per la prima volta aperto alla
possibilit\'a di impiegare il \textit{parallel computing} per
l'esecuzione di simulazioni numeriche \cite{Wilson1994}.

\subsection{Alcune applicazioni del calcolo parallelo}
Oggi permangono applicazioni in ambito scientifico
che possono essere eseguite
solo su \textit{cluster} di elaboratori oppure che richiedono lo sviluppo di architetture specifiche di dominio (DSA, \textit{Domain Specific Architecture}), considerate le loro caratteristiche \textit{compute-intensive}.\newline
Esempi di settori che hanno beneficiato dello sviluppo di
architetture innovative per il calcolo parallelo sono la
bioinformatica, l'elaborazione di immagini e video
e il settore aerospaziale, che ha potuto contare su simulazioni
numeriche sempre pi\'u accurate.

La rivoluzione introdotta dal calcolo parallelo non si limita esclusivamente al campo scientifico: un dominio applicativo che negli ultimi due decenni ha registrato uno sviluppo senza precedenti \'e l'intelligenza artificiale (AI, \textit{Artificial Intelligence}) e, in particolare, l'addestramento di modelli di AI mediante tecniche di \textit{Machine Learning}. \newline
I successi e le evoluzioni ottenuti in questo settore, e tangibili in diversi campi di applicazione come il riconoscimento di oggetti o l'industria della traduzione, non sarebbero stati fattibili se non supportati da sistemi di calcolo sufficientemente potenti e in grado di eseguire le operazioni aritmetiche richieste dallo svolgimento di compiti sempre pi\'u complessi.

Come ulteriore esempio, possiamo citare i calcolatori dei moderni centri di calcolo, chiamati \textit{Warehouse Scale Computer} (WSC), che costituiscono l'infrastruttura di erogazione dei moderni servizi Internet utilizzati ogni giorno da milioni di utenti, come i motori di ricerca, i \textit{social network} e i servizi di \textit{e-commerce}.\newline
Inoltre, la rivoluzione del \textit{cloud computing}, ovvero l'offerta via Internet di risorse di elaborazione \enquote{\textit{as a service}}, consente l'accesso ai WSC a chiunque sia dotato di una carta di credito.
\subsection{La barriera dell'energia}
\nocite{Spirito2021}
Il fattore fondamentale dietro all'adozione di massa delle architetture multiprocessore \'e la riduzione del consumo di energia elettrica offerta dai sistemi di calcolo paralleli; infatti, l'alimentazione e il raffreddamento delle centinaia di server presenti in un centro di calcolo moderno costituiscono una componente di costo non trascurabile, che risente solo marginalmente della disponibilit\'a di sistemi di raffreddamento dei microprocessori adatti a dissipare una grande quantit\'a di energia.

Il consumo di energia elettrica dei microprocessori viene misurato in Joule (\si{J}) ed \'e quasi interamente rappresentato dalla dissipazione di energia dinamica da parte dei transistori CMOS (\textit{Complementary Metal Oxide Semiconductor}), essendo la tecnologia dominante impiegata nella realizzazione dei moderni circuiti integrati.
Un transistore assorbe prevalentemente energia elettrica durante la commutazione alto-basso-alto del suo stato di uscita, secondo la formula
$$
    E = V^{+2} \cdot C_{L}
$$
dove $E$ rappresenta l'energia dissipata nelle due transizioni, $V^{+}$ la tensione di alimentazione e $C_{L}$ la capacit\'a di carico del transistore.\newline
La potenza dissipata $P_{D}$, assumendo che la frequenza di commutazione dello stato del transistore sia pari a $f$, \'e quindi data da
$$
    P_{D} = f \cdot E = f \cdot C_{L} \cdot V^{+2} \propto f_{C}
$$
dove $f_{C}$ \'e la frequenza di \textit{clock} del circuito, esprimibile in funzione di $f$.

In passato, i progettisti di circuiti integrati hanno tentato di contenere l'assorbimento di energia da parte dei microprocessori riducendo la tensione di alimentazione $V^{+}$ di circa il $15\%$ ad ogni nuova generazione di CPU, fino al raggiungimento del limite inferiore di 1 $\si{V}$.\newline
Al contempo, la diminuzione della tensione di alimentazione ha favorito la crescita delle correnti di dispersioni interne al transistore, tanto che nel 2008 circa il $40\%$ della potenza assorbita da un transistore era imputabile alle correnti di dispersione; ci si \'e imbattuti in una vera e propria \enquote{barriera dell'energia}.

In figura \ref{fig:PrestazioniProcessori}, possiamo notare come fino alla prima met\'a degli anni Ottanta del secolo scorso, la crescita annua delle prestazioni dei processori si attestava al $25\%$, per poi passare al $52\%$ grazie a importanti innovazioni nella progettazione e nell'organizzazione dei calcolatori; infine dal 2002, si sta continuando a registrare una crescita delle prestazioni meno evidente, pari al $3.5\%$ annuo, a causa del raggiungimento dei limiti relativi la potenza assorbita.\newline
La presenza di queste limitazioni tecnologiche ha accelerato la ricerca di nuove architetture per microprocessori, culminata con lo sviluppo del primo processore \textit{multicore}, IBM Power4, nel 2001 e il successivo lancio delle prime CPU \textit{multicore} destinate al mercato \textit{consumer}, nel 2006 da parte di Intel e AMD.
\begin{figure}[htbp]
    \centering
    \includegraphics[width=0.8\textwidth]{../Immagini/Capitolo 1/PrestazioniProcessori}
    \caption{Crescita nelle prestazioni dei processori dal 1978 al 2018; il grafico riporta le prestazioni dei processori, paragonandoli al VAX11/780 mediante l'esecuzione dei \textit{benchmark} SPECint \small{\textit{(Da J.L. Hennessy, D.A. Patterson, Computer Architecture: A quantitative Approach. Ed. 6. Waltham, MA:Elsevier, 2017)}}}
    \label{fig:PrestazioniProcessori}
\end{figure}

In futuro, il miglioramento delle prestazioni dei microprocessori sar\'a verosimilmente apportato dall'aumento del numero di \textit{core} montati su un singolo \textit{chip} piuttosto che dalla crescita della frequenza di clock dei singoli processori.
% Paragrafo 1.3: Le sfide nella progettazione di software parallelo
\section{Le sfide nella progettazione di software parallelo}
\label{par1.3}
Una caratteristica fondamentale posseduta da ogni programma a esecuzione parallela
è la scalabilit\'a, ovvero la capacit\'a di un sistema software di incrementare le proprie prestazioni in funzione della potenza di calcolo richiesta in un preciso istante \cite{Michael2007},
che consente di ottenere architetture tolleranti ai guasti assieme a un'elevata disponibilit\'a del sistema.

% Parte Finale della tesi
\backmatter

%Bibliografia
\bibliographystyle{IEEEtrans}
\addcontentsline{toc}{chapter}{Bibliografia}
\bibliography{./Bibliografia/Bibliografia.bib}

\end{document}