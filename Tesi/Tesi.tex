\documentclass[
	a4paper,
	twoside,
	12pt
]{book}

\usepackage{adjustbox}
\usepackage{amsmath}
\usepackage{amsthm}
\usepackage[italian]{babel}
\usepackage{bm} 
\usepackage{caption}
\usepackage[style=italian]{csquotes}
\usepackage{diagbox}
\usepackage{enumitem}
\usepackage{fancyhdr}
\usepackage[T1]{fontenc}
\usepackage{frontespizio}
\usepackage[top=3cm, bottom=3cm, left=3cm, right=3cm]{geometry}
\usepackage{graphicx}
\usepackage{parskip}
\usepackage{placeins}
\usepackage{setspace}
\usepackage{siunitx} 
\usepackage{todonotes}
\usepackage{xcolor}
\usepackage[colorlinks=false,linkbordercolor=blue,urlbordercolor=blue,citebordercolor=blue]{hyperref}
\usepackage{listings}

\newtheoremstyle{StileEsempio}
  {\topsep}     % Spazio sopra
  {\topsep}     % Spazio sotto
  {\normalfont} % Font del corpo del testo
  {}            % Indentazione
  {\bfseries}   % Font dell'intestazione
  {}            % Indentazione
  {\newline}    % A capo dopo il titolo
  {}            % Specifica dell'intestazione
\theoremstyle{StileEsempio}
\newtheorem{esempio}{Esempio}[chapter]

\graphicspath{{./Immagini}}
\onehalfspacing

\pagestyle{fancy}
\fancyhf{}
\fancyfoot[LE,RO]{\thepage}
\renewcommand{\headrulewidth}{0pt}
\fancypagestyle{plain}{%
  \fancyhf{}%
  \fancyfoot[LE,RO]{\thepage}%
  \renewcommand{\headrulewidth}{0pt}%
}

\hyphenation{MATLAB}

\definecolor{codegreen}{rgb}{0,0.6,0}
\definecolor{codegray}{rgb}{0.5,0.5,0.5}
\definecolor{codepurple}{rgb}{0.58,0,0.82}
\definecolor{backcolour}{rgb}{0.95,0.95,0.95}

\lstdefinestyle{MatlabStyle}{
    language=Matlab,
    backgroundcolor=\color{backcolour},   
    commentstyle=\color{codegreen},
    keywordstyle=\color{blue},
    numberstyle=\tiny\color{codegray},
    stringstyle=\color{codepurple},
    basicstyle=\ttfamily\footnotesize,
    breakatwhitespace=false,         
    breaklines=true,                 
    captionpos=b,                    
    keepspaces=true,                 
    numbers=left,                    
    numbersep=5pt,                  
    showspaces=false,                
    showstringspaces=false,
    showtabs=false,                  
    tabsize=2,
    frame=single,
    framerule=0pt,
    framesep=5pt,
    framexleftmargin=2pt
}
\lstset{style=MatlabStyle}

\setlength{\parindent}{0pt}
\setlength{\parskip}{1em}
\raggedbottom

\begin{document}
%Inizio Introduzione Tesi
\frontmatter
%Frontespizio
\begin{frontespizio}
	\Margini{3cm}{3cm}{3cm}{3cm}
	\Universita{Bergamo}
	\Logo[43.332mm]{./Immagini/Frontespizio/logo_unibg.pdf}
	\Divisione{Scuola di Ingegneria}
	\Corso[Laurea Triennale]{Ingegneria Informatica\\Classe n. L-8 Ingegneria dell’Informazione (D.M. 270/04)}
	\Titolo{Introduzione al calcolo parallelo in MATLAB\textsuperscript{\textregistered}}
	\Candidato[1086063]{Thomas Fabbris}
	\Relatore{Chiar.mo Prof.\ Fabio Previdi}
	\Annoaccademico{2024--2025}

	\begin{Preambolo*}
		\usepackage[italian]{babel}
		\usepackage[T1]{fontenc}
		\usepackage[utf8]{inputenc}
		\usepackage{microtype}
		\usepackage{lmodern}
		\usepackage{bm}

		\renewcommand{\frontinstitutionfont}{\fontsize{14}{17}\bfseries\scshape}
		\renewcommand{\fronttitlefont}{\fontsize{17}{21}\bfseries\scshape}
		\renewcommand{\frontfootfont}{\fontsize{12}{14}\bfseries\scshape}
	\end{Preambolo*}
\end{frontespizio}
% Indice
\tableofcontents
\mainmatter
% Introduzione della tesi
\chapter*{Introduzione}
\addcontentsline{toc}{chapter}{Introduzione}
%!TeX root = ../../Tesi.tex
L'obiettivo di questo capitolo \`e esibire una definizione puntuale di calcolo parallelo, un termine impiegato nel mondo
dell'HPC (\textit{High Performance Computing}) per riferirsi all’uso simultaneo di molteplici risorse di calcolo, consentendo la risoluzione di problemi a
elevata intensit\'a computazionale in tempi ragionevolmente brevi.

In seguito, investigheremo le cause che portarono alla nascita del parallelismo e descriveremo le principali difficolt\`a incontrate dai programmatori di
applicazioni durante l'implementazione di programmi a esecuzione parallela.
% Capitolo 1 (Calcolo parallelo: sfida o opportunità?)
\chapter{Calcolo parallelo: sfida o opportunit\'a?}
\label{cap1}
% Introduzione Capitolo 1
L'obiettivo di questo capitolo \`e esibire una definizione puntuale di calcolo parallelo, un termine impiegato nel mondo
dell'HPC (\textit{High Performance Computing}) per riferirsi all’uso simultaneo di molteplici risorse di calcolo, consentendo la risoluzione di problemi a 
elevata intensit\'a computazionale in tempi ragionevolmente brevi.

In seguito, investigheremo le cause che portarono alla nascita del parallelismo e descriveremo le principali difficolt\`a incontrate dai programmatori di 
applicazioni durante l'implementazione di programmi a esecuzione parallela.
% Paragrafo 1.1: Introduzione al calcolo parallelo
\section{Introduzione al calcolo parallelo}
\label{par1.1}
L'idea alla base del calcolo parallelo \'e che gli utenti di un qualsiasi sistema di elaborazione possono avere a disposizione tanti processori
quanti ne desiderano, per poi interconnetterli a formare un sistema
multiprocessore, le cui prestazioni sono, con buona approssimazione,
proporzionali al numero di processori montati.

La sostituzione di un singolo processore caratterizzato da un'elevata
capacit\'a di calcolo, tipicamente presente nelle architetture di sistemi
\textit{mainframe}, con un insieme di processori pi\'u efficienti
dal punto di vista energetico permette di raggiungere migliori prestazioni
per unit\'a di energia, a condizione che i programmi eseguiti siano
appositamente progettati per sfruttare pienamente la potenza di calcolo di ogni
singolo processore; approfondiremo questi aspetti nei paragrafi \ref{par1.2} e \ref{par1.3}.

Una tendenza introdotta da IBM nel 2001 nell'ambito della progettazione di sistemi paralleli \cite{tendler2001power4} è il raggruppamento
di diverse unit\'a di calcolo all'interno di un singolo circuito integrato; in questo contesto, i processori montati su un singolo \textit{chip} di silicio vengono chiamati \textit{core}.
Il microprocessore \textit{multicore} risultante appare al sistema operativo in esecuzione sull'elaboratore come l'insieme di $N$ processori, ognuno dei quali dotato di un set di registri e di una memoria \textit{cache} dedicati; solitamente si tratta di sistemi multiprocessore a memoria condivisa, in cui i \textit{core} condividono lo stesso spazio di indirizzamento fisico.\newline
Il funzionamento di questa categoria di sistemi multiprocessore si basa sul parallelismo a livello di attività (o a livello di processo): più
processori sono impiegati per svolgere diverse attivit\'a simultaneamente e ciascuna attivit\'a corrisponde ad applicazioni a singolo
\textit{thread}. In generale, ogni \textit{thread} esegue un'operazione ben definita e \textit{thread} differenti possono agire sugli stessi
dati o su insiemi di dati diversi.

D'altro canto, tutte le applicazioni che richiedono un utilizzo intensivo di risorse computazionali, oggi diffuse non pi\'u esclusivamente in ambito
scientifico ma anche in settori come quello dei motori di ricerca o dell'\textit{hosting} di siti Web, possono essere eseguite solo su \textit{cluster} di elaboratori, una tipologia di sistemi multiprocessore che si differenzia dai microprocessori \textit{multicore} per il fatto di essere costituita da un insieme di calcolatori completi, chiamati nodi, collegati tra loro per mezzo di una rete LAN (\textit{Local Area Network}).\newline
In ogni caso, il funzionamento di un sistema di elaborazione parallelo si basa sull'uso congiunto di processori diversi.\newline
Per sfruttare al meglio le potenzialit\'a offerte dai \textit{cluster} di elaboratori, i programmatori di applicazioni devono sviluppare programmi a esecuzione parallela efficienti e scalabili a seconda del numero di processori a disposizione durante l'esecuzione; risulta necessario applicare il parallelismo a livello di dati, un approccio che prevede la distribuzione di un insieme di dati di partenza sulle CPU del \textit{cluster} per poi eseguire la medesima operazione, ma su un insieme di dati diverso, su ogni processore.

Una tipica operazione parallelizzabile a livello di dati è la somma vettoriale poichè ogni componente del vettore risultante viene ottenuta
sommando le corrispondenti componenti omologhe dei due vettori di partenza, in modo indipendente dalle altre; possiamo sin da subito intuire che una condizione necessaria per la parallelizzazione di un qualsiasi algoritmo è l'indipendenza tra le operazioni eseguite in un certo passo.
Ad esempio, supponiamo di dover sommare due vettori di numeri reali di dimensione $N$ avvalendoci di un sistema \textit{dual-core}, ossia di un sistema parallelo dotato di un microprocessore con
due \text{core} distinti. Sarebbe conveniente avviare un thread separato su ogni \textit{core} specializzato nella somma di due componenti omologhe dei vettori operandi; attraverso una opulata distribuzione dei dati, otterremmo che il \textit{thread} sul primo \textit{core} somma le componenti omologhe dei vettori di partenza
da $1$ a $\left\lfloor\frac{N}{2}\right\rfloor$, mentre il secondo \textit{core} si occuperebbe della somma delle componenti da $\left\lceil\frac{N}{2}\right\rceil$ a $N$.

In realtà, la rigida classificazione proposta tra parallelismo a livello di attivit\'a e parallelismo a livello di dati non ritrova un diretto
riscontro nella realt\'a, in quanto sono stati sviluppati programmi che sfruttano entrambi gli approcci al fine di massimizzare le prestazioni.

Cogliamo l'occasione per precisare la terminologia, in parte gi\'a utilizzata, impiegata per descrivere la componente hardware e la componente software di un sistema di elaborazione: l'hardware, descrivendo con questo termine il processore del sistema, pu\'o essere classificato in seriale, come nel caso di un processore \textit{single core}, o parallelo, come nel caso di un processore \textit{multicore}, mentre il software viene detto sequenziale o concorrente, a seconda della presenza o meno di diversi processi cooperanti la cui esecuzione \'e influenzata dagli altri processi presenti nel sistema.\newline
Naturalmente, un programma concorrente pu\'o essere eseguito sia su hardware seriale che su hardware parallelo.\newline
Infine, con il termine programma ad esecuzione parallela, o pi\'u semplicemente software parallelo, indicheremo un programma (sequenziale o concorrente) eseguito su hardware parallelo.
% Paragrafo 1.2: Le cause a supporto del parallelismo
\section{Le cause a supporto del parallelismo}
\label{par1.2}
L'attenzione riservata all'elaborazione parallela da parte della comunit\'a
scientifica risale al 1957, anno in cui la
Compagnie des Machines Bull (l'odierna Bull SAS) ha annunciato Gamma 60, un computer \textit{mainframe}
equipaggiato con la prima architettura della storia in grado di offrire un supporto diretto
al parallelismo, mentre l'anno successivo, i ricercatori IBM John
Cocke e Daniel Slotnick hanno per la prima volta aperto alla
possibilit\'a di impiegare il \textit{parallel computing} per
l'esecuzione di simulazioni numeriche \cite{Wilson1994}.

\subsection{Alcune applicazioni del calcolo parallelo}
Oggi permangono applicazioni in ambito scientifico
che possono essere eseguite
solo su \textit{cluster} di elaboratori oppure che richiedono lo sviluppo di architetture specifiche di dominio (DSA, \textit{Domain Specific Architecture}), considerate le loro caratteristiche \textit{compute-intensive}.\newline
Esempi di settori che hanno beneficiato dello sviluppo di
architetture innovative per il calcolo parallelo sono la
bioinformatica, l'elaborazione di immagini e video
e il settore aerospaziale, che ha potuto contare su simulazioni
numeriche sempre pi\'u accurate.

La rivoluzione introdotta dal calcolo parallelo non si limita esclusivamente al campo scientifico: un dominio applicativo che negli ultimi due decenni ha registrato uno sviluppo senza precedenti \'e l'intelligenza artificiale (AI, \textit{Artificial Intelligence}) e, in particolare, l'addestramento di modelli di AI mediante tecniche di \textit{Machine Learning}. \newline
I successi e le evoluzioni ottenuti in questo settore, e tangibili in diversi campi di applicazione come il riconoscimento di oggetti o l'industria della traduzione, non sarebbero stati fattibili se non supportati da sistemi di calcolo sufficientemente potenti e in grado di eseguire le operazioni aritmetiche richieste dallo svolgimento di compiti sempre pi\'u complessi.

Come ulteriore esempio, possiamo citare i calcolatori dei moderni centri di calcolo, chiamati \textit{Warehouse Scale Computer} (WSC), che costituiscono l'infrastruttura di erogazione dei moderni servizi Internet utilizzati ogni giorno da milioni di utenti, come i motori di ricerca, i \textit{social network} e i servizi di \textit{e-commerce}.\newline
Inoltre, la rivoluzione del \textit{cloud computing}, ovvero l'offerta via Internet di risorse di elaborazione \enquote{\textit{as a service}}, consente l'accesso ai WSC a chiunque sia dotato di una carta di credito.
\subsection{La barriera dell'energia}
\nocite{Spirito2021}
Il fattore fondamentale dietro all'adozione di massa delle architetture multiprocessore \'e la riduzione del consumo di energia elettrica offerta dai sistemi di calcolo paralleli; infatti, l'alimentazione e il raffreddamento delle centinaia di server presenti in un centro di calcolo moderno costituiscono una componente di costo non trascurabile, che risente solo marginalmente della disponibilit\'a di sistemi di raffreddamento dei microprocessori adatti a dissipare una grande quantit\'a di energia.

Il consumo di energia elettrica dei microprocessori viene misurato in Joule (\si{J}) ed \'e quasi interamente rappresentato dalla dissipazione di energia dinamica da parte dei transistori CMOS (\textit{Complementary Metal Oxide Semiconductor}), essendo la tecnologia dominante impiegata nella realizzazione dei moderni circuiti integrati.
Un transistore assorbe prevalentemente energia elettrica durante la commutazione alto-basso-alto del suo stato di uscita, secondo la formula
$$
    E = V^{+2} \cdot C_{L}
$$
dove $E$ rappresenta l'energia dissipata nelle due transizioni, $V^{+}$ la tensione di alimentazione e $C_{L}$ la capacit\'a di carico del transistore.\newline
La potenza dissipata $P_{D}$, assumendo che la frequenza di commutazione dello stato del transistore sia pari a $f$, \'e quindi data da
$$
    P_{D} = f \cdot E = f \cdot C_{L} \cdot V^{+2} \propto f_{C}
$$
dove $f_{C}$ \'e la frequenza di \textit{clock} del circuito, esprimibile in funzione di $f$.

In passato, i progettisti di circuiti integrati hanno tentato di contenere l'assorbimento di energia da parte dei microprocessori riducendo la tensione di alimentazione $V^{+}$ di circa il $15\%$ ad ogni nuova generazione di CPU, fino al raggiungimento del limite inferiore di 1 $\si{V}$.\newline
Al contempo, la diminuzione della tensione di alimentazione ha favorito la crescita delle correnti di dispersioni interne al transistore, tanto che nel 2008 circa il $40\%$ della potenza assorbita da un transistore era imputabile alle correnti di dispersione; ci si \'e imbattuti in una vera e propria \enquote{barriera dell'energia}.

In figura \ref{fig:PrestazioniProcessori}, possiamo notare come fino alla prima met\'a degli anni Ottanta del secolo scorso, la crescita annua delle prestazioni dei processori si attestava al $25\%$, per poi passare al $52\%$ grazie a importanti innovazioni nella progettazione e nell'organizzazione dei calcolatori; infine dal 2002, si sta continuando a registrare una crescita delle prestazioni meno evidente, pari al $3.5\%$ annuo, a causa del raggiungimento dei limiti relativi la potenza assorbita.\newline
La presenza di queste limitazioni tecnologiche ha accelerato la ricerca di nuove architetture per microprocessori, culminata con lo sviluppo del primo processore \textit{multicore}, IBM Power4, nel 2001 e il successivo lancio delle prime CPU \textit{multicore} destinate al mercato \textit{consumer}, nel 2006 da parte di Intel e AMD.
\begin{figure}[htbp]
    \centering
    \includegraphics[width=0.8\textwidth]{../Immagini/Capitolo 1/PrestazioniProcessori}
    \caption{Crescita nelle prestazioni dei processori dal 1978 al 2018; il grafico riporta le prestazioni dei processori, paragonandoli al VAX11/780 mediante l'esecuzione dei \textit{benchmark} SPECint \small{\textit{(Da J.L. Hennessy, D.A. Patterson, Computer Architecture: A quantitative Approach. Ed. 6. Waltham, MA:Elsevier, 2017)}}}
    \label{fig:PrestazioniProcessori}
\end{figure}

In futuro, il miglioramento delle prestazioni dei microprocessori sar\'a verosimilmente apportato dall'aumento del numero di \textit{core} montati su un singolo \textit{chip} piuttosto che dalla crescita della frequenza di clock dei singoli processori.
% Paragrafo 1.3: Le sfide nella progettazione di software parallelo
\section{Le sfide nella progettazione di software parallelo}
\label{par1.3}
Una caratteristica fondamentale posseduta da ogni programma a esecuzione parallela
è la scalabilit\'a, ovvero la capacit\'a di un sistema software di incrementare le proprie prestazioni in funzione della potenza di calcolo richiesta in un preciso istante \cite{Michael2007},
che consente di ottenere architetture tolleranti ai guasti assieme a un'elevata disponibilit\'a del sistema.

% Parte Finale della tesi
% Capitolo 2 (Un linguaggio per il calcolo parallelo: MATLAB)
\chapter{Un linguaggio per il calcolo parallelo: MATLAB}
\label{cap2}
% Introduzione Capitolo 2
L'accesso diffuso a sistemi di elaborazione multiprocessore ha aumentato 
la domanda di mercato per soluzioni software a supporto dello sviluppo di programmi a esecuzione parallela. \newline 
Da questo processo neanche MATLAB (abbreviazione di \textit{Matrix Laboratory}), un ambiente di programmazione proprietario per il calcolo scientifico, ne \`e esente, 
creando tutte le condizioni necessarie per l'aggiunta di nuove funzionalit\`a dedicate all'elaborazione parallela.

L'obiettivo di questo capitolo \`e fornire una panoramica delle funzionalit\`a salienti di MATLAB per il calcolo parallelo, 
con un particolare focus sulle motivazioni e sulle scelte di design che hanno guidato l'implementazione di questa nuova parte del linguaggio.
% Paragrafo 2.1: Gli ingredienti per un MATLAB parallelo
\section{Gli ingredienti per un MATLAB parallelo}
\label{par2.1}
%!TeX root = ../../Tesi.tex
\nocite{Sharma2009}
\subsection{Una breve prospettiva storica}
L'approccio seguito dai progettisti di The MathWorks\footnote{La \textit{software house} statunitense, con sede in Massachusetts (Stati Uniti), che si occupa dello sviluppo di MATLAB e di altri prodotti per il calcolo scientifico.}
per estendere MATLAB  al mondo del calcolo parallelo \`e stato modificare le caratteristiche del linguaggio 
stesso, cominciando dall'aggiunta di \textit{routine} comunemente impiegate nella risoluzione di problemi \textit{embarrassingly parallel}. 

A partire dai primi passi compiuti in questa direzione negli anni Ottanta del secolo scorso da Cleve Moler, l'autore del linguaggio, ci si imbatt\'e
nel fatto che il modello di memoria globale di MATLAB, secondo il quale le variabili definite dall'utente e importate dall'esterno vengono conservate 
in un'area di memoria allocata dalla sessione attiva di MATLAB, era in contrasto con il modello di memoria condivisa impiegato dalla maggioranza dei sistemi 
multiprocessore.

Questa incompatibilit\`a causò dei rallentamenti al progetto di parallelizzazione di MATLAB, ma le pressioni esterne di coloro che auspicavano a un suo completamento erano troppo insistenti per essere ignorate.\newline
La crescente disponibilit\`a di sistemi multiprocessore aveva reso il calcolo parallelo un argomento presente sulla bocca di tutti gli specialisti del 
settore; la comparsa delle prime architetture \textit{multicore} e l'ascesa dei \textit{cluster} Beowulf\footnote{I \textit{cluster} Beowulf sono \textit{cluster} costituiti dall'interconessione di componenti hardware commerciali, ad esempio PC al termine della loro vita utile, mediante una tradizionale rete LAN. } permisero una diffusione massiccia dei sistemi di calcolo ad alte prestazioni, che precedette la \enquote{democratizzazione} dei WSC portata dal \textit{cloud computing}.\newline   
Inoltre, MATLAB era gi\`a allora un ambiente di programmazione affermato all'interno della comunit\`a scientifica e quindi doveva fornire ai propri utenti un prodotto completo e funzionale 
in tutti gli scenari applicativi, inclusi i progetti a elevata intensit\`a computazionale.

Ecco che nel novembre del 2004 vennero rilasciati al pubblico i primi risultati di questo progetto, sotto le vesti di due pacchetti software addizionali (chiamati \textit{toolbox} nel vocabolario tecnico del linguaggio): il Distributed Computing 
Toolbox\textsuperscript{\texttrademark} e il MATLAB Distributed Computing Engine\textsuperscript{\texttrademark}\footnote{I due nomi commerciali sono i corrispettivi degli odierni Parallel 
Computing Toolbox\textsuperscript{\texttrademark} e MATLAB Parallel Server\textsuperscript{\texttrademark}.}.

\subsection{Gli aspetti imprescindibili dell'implementazione}
\label{sec2.1.2}
L'adattamento di MATLAB al calcolo parallelo non fu condotto in modo casuale, bens\'i le aggiunte al linguaggio furono ponderate attentamente a partire dalle informazioni ricavate dai sondaggi condotti nelle fasi preliminari del progetto.\newline
Per questo motivo, il modello di programmazione proposto da MATLAB \`e adatto all'esecuzione di programmi paralleli su sistemi \textit{multicore} e su \textit{cluster} di elaboratori, trattandosi delle architetture di calcolo parallelo pi\`u comuni in ambito industriale.

Di seguito elenchiamo, in ordine decrescente di importanza, gli obiettivi di progettazione che ispirarono e continuano a ispirare il processo di parallelizzazione di MATLAB;
\begin{itemize}
    \item la programmabilit\`a, cio\'e la capacit\`a di creare programmi che soddisfino i requisiti degli utenti e che siano facili da mantenere per gli sviluppatori;
    \item l'esecuzione di codice arbitrario sui sistemi multiprocessore supportati;
    \item l'astrazione da dettagli irrilevanti durante l'implementazione; di conseguenza, lo sviluppatore medio non deve pi\`u preoccuparsi di aspetti quali lo \textit{scheduling} delle \textit{task} e la distribuzione dei dati alle unit\`a di lavoro, in quanto vengono gestiti automaticamente dal linguaggio;
    \item l'indipendenza del programma dall'allocazione delle risorse computazionali: un software parallelo scritto in MATLAB deve funzionare correttamente sia quando eseguito su un sistema multiprocessore che su un sistema monoprocessore, adattando il suo comportamento alle risorse di calcolo disponibili;
    \item l'accesso a costrutti di programmazione di prima classe\footnote{Secondo la classificazione proposta dall'informatico britannico Christopher Strachey \cite{SICP96}, i costrutti di programmazione di prima classe possono essere manipolati liberamente nelle istruzioni del linguaggio; in pratica, devono poter essere passati come parametri attuali durante l'invocazione di una procedura, restituiti come valore di ritorno di una funzione e assegnati a variabili o a strutture dati.}. 
\end{itemize}
Il percorso di trasformazione di MATLAB al fine di renderlo appetibile al calcolo parallelo non \`e ancora giunto al termine. \newline La destinazione finale fissata 
dagli addetti ai lavori \`e la realizzazione del modello di linguaggio ideato dal direttore tecnico di TheMathWorks, Roy Lurie \cite{Lurie2007},
secondo cui gli esperti di dominio inseriscono annotazioni minimali al codice sorgente per esprimere l'intenzione di eseguire il programma 
su pi\`u processori simultaneamente.

% Paragrafo 2.2: Parallel Computing Toolbox
\section{Parallel Computing Toolbox}
\label{par2.2}
\nocite{MathWorksParallelComputing}
Il \textit{Parallel Computing Toolbox}, spesso abbreviato in PCT, permette di risolvere problemi \textit{data-intensive} e \textit{compute-intesive} sfruttando 
la potenza di calcolo offerta dai microprocessori \textit{multicore} e dai moderni cluster di \textit{elaboratori}. \newline
Costrutti di programmazione di alto livello, come i vettori distribuiti, consentono di sviluppare applicazioni MATLAB scalabili senza ricorrere alla programmazione MPI \footnote{La \textit{Message 
Passing Interface}, o semplicemente MPI, rappresenta lo standard per il modello di comunicazione interprocesso, basato sullo scambio di messaggi, impiegato nelle elaborazioni 
parallele su sistemi distribuiti \cite{NMSUMPIIntro}}.\newline
Inoltre, la stessa applicazione pu\`o essere eseguita su \textit{cluster} o su server in \textit{cloud} senza apportare alcuna modifica al codice grazie a MATLAB 
\textit{Parallel Server}, cos\`i da concentrarsi esclusivamente sullo sviluppo del modello matematico migliore per il caso d'uso in questione.

Incominciamo il nostro studio del PCT riportando alcune definizioni di particolari aspetti del modello di programmazione di MATLAB considerate fondamentali per la prosecuzione della trattazione.
\begin{itemize}
\item \textit{Client}: termine impiegato per identificare la sessione di MATLAB attiva con cui l'utente finale sta interagendo; tipicamente, corrisponde con il \textit{computer} usato dallo sviluppatore durante la prototipazione e lo sviluppo in locale del programma.\newline
Attraverso le funzionalit\`a offerte dal PCT, un \textit{client} pu\`o gestire la computazione da eseguire suddividendola in  task pi\`u semplici e assegnando ciascuna task a un MATLAB \textit{worker}.
\item \textit{Worker}: corrisponde a un'istanza di MATLAB, priva di interfaccia grafica, controllata da un \textit{client} e in grado di fornire la potenza del motore di calcolo del linguaggio.
\item \textit{Parallel Pool}: spesso abbreviato in parpool, \`e un insieme di \textit{worker} comunicanti che possono eseguire codice interattivamente.
\end{itemize}

Una prima distinzione da sottolineare \`e quella tra l'infrastruttura e i componenti del linguaggio esposti dagli strumenti di calcolo parallelo in MATLAB. 
Il linguaggio comprende costrutti di programmazione paralleli e funzioni con supporto automatico al parallelismo mentre l'infrastruttura riguarda i meccanismi 
a supporto del linguaggio, come il protocollo seguito per il trasferimento del codice e dei dati alle unit\`a di lavoro del sistema. \newline
Nelle prossime sezioni, esamineremo da vicino alcuni costrutti paralleli offerti da MATLAB, accennando solamente all'infrastruttura sottostante, nonostante entrambe le componenti siano imprescindibili all'interno del \textit{framework} in questione.

L'architettura di riferimento fino alla fine del capitolo \`e schematizzata in figura \ref{fig:ArchitetturaRiferimento}.\newline
MATLAB \textit{Parallel Server} comprende un insieme di \textit{worker}, in esecuzione sui nodi del \textit{cluster}, che ricevono le 
\textit{task} computazionali dal \textit{client} attraverso specifiche funzioni del \textit{Parallel Computing Toolbox}. \newline
I \textit{worker} prelevano il codice da eseguire e i dati su cui lavorare da una memoria di massa condivisa popolata dall'\textit{head node} (non rappresentato in figura), un nodo speciale eletto all'interno del \textit{cluster} che si occupa dell'assegnazione delle attivit\`a ai \textit{worker} e dell'interfacciamento con il \textit{client}.\newline
Una volta terminata l'elaborazione, i risultati vengono raccolti dal nodo \textit{master} e trasferiti all'interno dello spazio di lavoro del \textit{client} mediante il canale di comunicazione instaurato tra il \textit{client} e MATLAB \textit{Parallel Server}.

\begin{figure}[htbp]
    \centering
    \includegraphics[width=0.8\textwidth]{../Immagini/Capitolo 2/ReferenceArchitecture.png}
    \caption{Architettura di riferimento per gli strumenti di calcolo parallelo in MATLAB. \small{(Da \url{https://it.mathworks.com/products/matlab-parallel-server.html})}}
    \label{fig:ArchitetturaRiferimento}
\end{figure}

A questo punto, accenniamo alle modalit\`a di esecuzione del software parallelo su un sistema multiprocessore supportate dall'ambiente MATLAB:
\begin{itemize}
    \item parallelizzazione implicita: alcune funzioni, se richiamate nel codice sorgente del programma, sfruttano le librerie di \textit{runtime} del linguaggio in modo da essere 
    eseguite su \textit{thread} distinti all'interno della stessa sessione, beneficiando di notevoli miglioramenti di \textit{performance} su sistemi con un numero elevato 
    di processori;
    \item parallelizzazione esplicita: il carico di lavoro del programma viene automaticamente suddiviso in \textit{task} elementari, ciascuna delle quali viene poi assegnata a un \textit{worker} per l'esecuzione.
\end{itemize}

\subsection{Il paradigma di programmazione parallela implicita}
I \textit{toolbox} di MATLAB sono dotati di un crescente numero di funzioni con supporto automatico al parallelismo, al fine di beneficiare di tutti 
i vantaggi propri dall'elaborazione parallela senza modificare i file di codice preesistenti, in accordo con i principi di design elencati nel paragrafo \ref{par2.1}. 

Alcune funzioni, come \lstinline|mldivide| impiegata per la risoluzione di sistemi di equazioni lineari, vengono eseguite in parallelo di \textit{default} se invocate dalla sessione principale di MATLAB. 

Ragionando sulla nostra architettura di riferimento, il \textit{multithreading} implicito viene attivato solo quando la funzione viene eseguita direttamente dal \textit{client}, mentre viene evitato se l'esecuzione \'e a carico dei nodi del \textit{cluster} per evitare un parallelismo \enquote{annidato}, che degraderebbe le prestazioni dell'intero sistema. \newline
In quest'ottica, possiamo notare come i progettisti del linguaggio abbiano pensato a un \textit{worker} come un'unit\`a di elaborazione a singolo \textit{thread}.

Il \textit{client}, quando incontra una funzione con supporto automatico al parallelismo nel codice sorgente del programma, avvia un \textit{parpool} per la sua esecuzione in parallelo. \newline
Un apposito profilo di configurazione determina le caratteristiche dell'ambiente di elaborazione parallela e, in particolare, PCT permette di scegliere tra i seguenti profili preimpostati:
\begin{itemize}
    \item \textit{Processes}: i \textit{worker} vengono attivati come processi indipendenti eseguiti dai \textit{core} fisici del calcolatore su cui \`e attiva la sessione principale di MATLAB.
    \item \textit{Threads}: i \textit{worker} sono ospitati da \textit{thread} e non pi\`u da processi veri e propri. I vantaggi portati da questo ambiente parallelo sono un minor uso di memoria, un basso costo di comunicazione tra i \textit{worker} e uno \textit{scheduling} delle attivit\`a particolarmente performante, a scapito della disponibilit\`a di una ristretta gamma di funzioni con supporto al parallelismo su \textit{thread}.
\end{itemize}
Per quanto riguarda la scelta del numero di \textit{worker} per l'ambiente \textit{Processes}, \'e consigliato riservare un motore di calcolo per ogni \textit{core} fisico disponibile, ignorando la presenza di eventuali \textit{core} virtuali; infatti, questi ultimi condividono alcune risorse di calcolo all'interno dello stesso processore, tra cui la \textit{Floating Point Unit} (FPU), e poich\'e la maggior parte delle elaborazioni in MATLAB richiede l'esecuzione di operazioni aritmetiche in virgola mobile, limitare a uno il numero di \textit{worker} per unit\`a di esecuzione pu\`o aumentare la stabilit\`a del sistema. \newline 
L'unica eccezione \`e rappresentata dalle applicazioni \textit{data-intensive}, per le quali potrebbe essere conveniente portare il numero di \textit{worker} per \textit{core} fisico a due.

In ogni caso, il massimo numero di \textit{worker} presenti in un singolo \textit{parpool} a supporto della parallelizzazione implicita \'e pari a 512, a prescindere dalle specifiche del calcolatore utilizzato.
\begin{figure}[htbp]
    \centering
    \includegraphics[width=0.8\textwidth]{../Immagini/Capitolo 2/ImplicitParallelization.png}
    \caption{Rappresentazione del modello di parallelizzazione implicita di MATLAB su un sistema \textit{dual-core} \small{(Da \url{https://it.mathworks.com/discovery/matlab-multicore.html})}}
    \label{fig:ParallelismoImplicito}
\end{figure}\newline
Se una funzione non include il supporto automatico al parallelimo, possiamo trasferire l'esecuzione del programma a una \textit{workstation}, in modo da beneficiare dello \textit{speedup} offerto da un sistema con maggiore capacit\`a di calcolo, oppure possiamo utilizzare il paradigma di programmazione parallela esplicita supportato dal \textit{Parallel Computing Toolbox}.

\subsection{Il paradigma di programmazione parallela esplicita}

Il modello di programmazione parallela esplicita in MATLAB conta sulla presenza di costrutti di programmazione parallela a diversi livelli di astrazione.
\backmatter

%Bibliografia
\bibliographystyle{IEEEtrans}
\addcontentsline{toc}{chapter}{Bibliografia}
\bibliography{./Bibliografia/Bibliografia.bib}

\end{document}