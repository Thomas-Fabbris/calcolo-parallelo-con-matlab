\documentclass[
	a4paper,
	twoside,
	12pt
]{book}

\usepackage[T1]{fontenc}
\usepackage[italian]{babel}
\usepackage[top=3cm, bottom=3cm, left=3cm, right=3cm]{geometry}
\usepackage{fancyhdr}
\usepackage{setspace}
\usepackage{graphicx}
\usepackage{frontespizio}
\usepackage{placeins}
\usepackage{adjustbox}
\usepackage{bm} 
\usepackage{amsmath}
\usepackage{todonotes}

\graphicspath{{./Immagini}}
\onehalfspacing

\pagestyle{fancy}
\fancyhf{}
\fancyfoot[LE,RO]{\thepage}
\renewcommand{\headrulewidth}{0pt}
\fancypagestyle{plain}{%
  \fancyhf{}%
  \fancyfoot[LE,RO]{\thepage}%
  \renewcommand{\headrulewidth}{0pt}%
}

\begin{document}
%Inizio Introduzione Tei
\frontmatter
%Frontespizio
\begin{frontespizio}
	\Margini{3cm}{3cm}{3cm}{3cm}
	\Universita{Bergamo}
	\Logo[43.332mm]{./Immagini/Frontespizio/logo_unibg.pdf}
	\Divisione{Scuola di Ingegneria}
	\Corso[Laurea Triennale]{Ingegneria Informatica\\Classe n. L-8 Ingegneria dell’Informazione (D.M. 270/04)}
	\Titolo{Introduzione al calcolo parallelo in MATLAB\textsuperscript{\textregistered}}
	\Candidato[1086063]{Thomas Fabbris}
	\Relatore{Chiar.mo Prof.\ Fabio Previdi}
	\Annoaccademico{2024--2025}

	\begin{Preambolo*}
		\usepackage[italian]{babel}
		\usepackage[T1]{fontenc}
		\usepackage[utf8]{inputenc}
		\usepackage{microtype}
		\usepackage{lmodern}
		\usepackage{bm}

		\renewcommand{\frontinstitutionfont}{\fontsize{14}{17}\bfseries\scshape}
		\renewcommand{\fronttitlefont}{\fontsize{17}{21}\bfseries\scshape}
		\renewcommand{\frontfootfont}{\fontsize{12}{14}\bfseries\scshape}
	\end{Preambolo*}
\end{frontespizio}

% Indice
\tableofcontents

\mainmatter

% Introduzione della tesi
\chapter*{Introduzione}
\addcontentsline{toc}{chapter}{Introduzione}
%!TeX root = ../../Tesi.tex
L'obiettivo di questo capitolo \`e esibire una definizione puntuale di calcolo parallelo, un termine impiegato nel mondo
dell'HPC (\textit{High Performance Computing}) per riferirsi all’uso simultaneo di molteplici risorse di calcolo, consentendo la risoluzione di problemi a
elevata intensit\'a computazionale in tempi ragionevolmente brevi.

In seguito, investigheremo le cause che portarono alla nascita del parallelismo e descriveremo le principali difficolt\`a incontrate dai programmatori di
applicazioni durante l'implementazione di programmi a esecuzione parallela.

% Capitolo 1 (Calcolo parallelo: sfida o opportunità?)
\chapter{Calcolo parallelo: sfida o opportunit\'a?}
\label{cap1}
L'obiettivo di questo capitolo \'e dare una definizione precisa di calcolo parallelo, un termine spesso impiegato all'interno del mondo
del supercalcolo per descrivere l’uso simultaneo di molteplici risorse di calcolo per la risoluzione di problemi ad elevata intensità computazionale
in tempi ragionevolmente brevi, tipici delle applicazioni sviluppate in ambito scientifico.

In seguito, andremo ad investigare le cause che hanno portato alla nascita del parallelismo e a descrivere le principali difficolt'\a tradizionalmente incontrate dai programmatori durante l'implementazione di programmi ad esecuzione parallela.

% Parte Finale della tesi
\backmatter

%Bibliografia
\bibliographystyle{IEEEtrans}
\addcontentsline{toc}{chapter}{Bibliografia}
\bibliography{./Bibliografia/Bibliografia.bib}

\end{document}