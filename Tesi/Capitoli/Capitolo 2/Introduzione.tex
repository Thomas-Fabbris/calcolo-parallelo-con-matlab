L'accesso diffuso a sistemi di elaborazione multiprocessore ha aumentato
la domanda di mercato per soluzioni software a supporto dello sviluppo di programmi a esecuzione parallela. \newline
Gli ambienti di programmazione per il calcolo scientifico, MATLAB (abbreviazione di \textit{Matrix Laboratory}) incluso, hanno recepito questa tendenza,
creando tutte le condizioni necessarie per l'aggiunta di nuove funzionalit\`a dedicate all'elaborazione parallela.

In questo capitolo forniremo una rapida panoramica delle funzionalit\`a di MATLAB per il calcolo parallelo,
con un particolare focus sulle motivazioni e sulle scelte di design che hanno guidato l'implementazione di questa nuova parte del linguaggio.