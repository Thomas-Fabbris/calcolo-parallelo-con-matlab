\nocite{Sharma2009}
\subsection{Una breve prospettiva storica}
L'approccio seguito dai progettisti di The MathWorks \footnote{La \textit{software house} americana, con sede in Massachusetts (Stati Uniti), che si occupa dello sviluppo di MATLAB e di altri prodotti per il calcolo scientifico.}
per introdurre in MATLAB le funzionalit\`a a supporto del calcolo parallelo \`e stato quello di modificare le caratteristiche del linguaggio 
stesso, partendo dalla scrittura di \textit{routine} specializzate nella risoluzione di problemi \textit{embarrassingly parallel}, per i quali il principale ostacolo da considerare
sappiamo essere la loro complessit\`a computazionale intrinseca. 

A partire dai primi passi compiuti in questa direzione negli anni Ottanta del secolo scorso da Cleve Moler, l'autore originale del linguaggio, ci si scontr\`o 
con il fatto che il modello di memoria globale tipico di MATLAB, secondo il quale le variabili definite dall'utente o importate dall'esterno vengono conservate 
in un'area di memoria allocata dalla sessione di MATLAB attiva, era in contrasto con il modello di memoria condivisa impiegato dalla maggioranza dei sistemi 
multiprocessore.

Questa difficoltà causò dei rallentamenti al progetto che mirava alla parallelizzazione di MATLAB, ma le pressioni esterne per il suo completamento erano troppo insistenti per essere ignorate.\newline
La crescente disponibilit\`a di sistemi multiprocessore aveva reso il calcolo parallelo un argomento presente sulla bocca di tutti gli specialisti del 
settore: l'apparizione delle prime architetture \textit{multicore} e la costruzione di \textit{cluster} Beowulf \footnote{I \textit{cluster} Beowulf sono \textit{cluster} costituiti dall'interconessione di prodotti hardware commerciali, ad esempio PC giunti al termine della loro vita utile, mediante una tradizionale rete LAN. } avevano permesso una notevole diffusione dei sistemi di calcolo ad alte prestazioni, anni prima della \enquote{democratizzazione} dei WSC portata dal \textit{cloud computing}.\newline   
Inoltre, MATLAB era gi\`a un ambiente di calcolo scientifico affermato e quindi doveva fornire alla propria comunit\`a di utenti un prodotto completo e funzionale 
in tutti gli scenari applicativi, inclusi i progetti a elevata intensit\`a computazionale.

Ecco che nel novembre del 2004 vennero resi disponibili al pubblico i primi risultati di questo progetto, sotto le vesti di due pacchetti software addizionali (chiamati \textit{toolbox} o \textit{add-on} secondo la terminologia impiegata dal linguaggio): il Distributed Computing 
Toolbox\textsuperscript{\texttrademark} e il MATLAB Distributed Computing Engine\textsuperscript{\texttrademark}\footnote{I due nomi commerciali sono i corrispettivi degli odierni Parallel 
Computing Toolbox\textsuperscript{\texttrademark} e MATLAB Parallel Server\textsuperscript{\texttrademark}.}.

\subsection{Gli aspetti imprescindibili dell'implementazione}
\label{sec2.1.2}
L'espansione di MATLAB al calcolo parallelo non fu condotta in modo casuale, ma le aggiunte al linguaggio furono ponderate attentamente a partire dalle informazioni ricavate da sondaggi condotti durante la fase di raccolta dei requisiti.\newline
Per questo motivo, il modello di programmazione parallela proposto da MATLAB \`e idoneo all'esecuzione di programmi paralleli su sistemi \textit{multicore} e su \textit{cluster} di elaboratori, trattandosi delle architetture di calcolo parallelo pi\`u comuni in ambito industriale.\newline
Di seguito elenchiamo, in ordine decrescente di importanza, gli obiettivi di design che hanno ispirato il processo di parallelizzazione di MATLAB:
\begin{itemize}
    \item la programmabilit\`a, cio\'e la capacit\`a di creare programmi che soddisfano i requisiti degli utenti e che siano facili da mantenere per gli sviluppatori;
    \item l'esecuzione di codice arbitrario sui sistemi multiprocessore supportati;
    \item l'astrazione da dettagli irrilevanti durante l'implementazione di programmi a esecuzione parallela; di conseguenza, lo sviluppatore medio non deve pi\`u preoccuparsi di aspetti come lo \textit{scheduling}, la sincronizzazione tra le \textit{task} e la distribuzione dei dati in input alle unit\`a di lavoro;
    \item l'indipendenza del programma dall'allocazione delle risorse computazionali: un software parallelo scritto in MATLAB deve funzionare correttamente sia quando viene eseguito su un sistema multiprocessore che su un sistema monoprocessore, adattandosi alle risorse di calcolo a disposizione durante l'esecuzione;
    \item l'accesso a costrutti di programmazione di prima classe \footnote{Secondo la classificazione proposta dall'informatico britannico Christopher Strachey \cite{SICP96}, i costrutti di programmazione di prima classe possono essere manipolati liberamente nelle istruzioni del linguaggio; in pratica, devono poter essere passati come parametri attuali durante l'invocazione di una procedura, restituiti come valori di ritorno di una funzione e assegnati a variabili o a strutture dati.}. 
\end{itemize}
Il percorso di trasformazione di MATLAB al fine di renderlo pi\`u appetibile al calcolo parallelo non \`e ancora giunto al termine. \newline La destinazione finale fissata 
dagli addetti ai lavori \`e la realizzazione del modello di linguaggio ideato dal direttore tecnico di TheMathWorks Roy Lurie \cite{Lurie2007}
secondo cui gli esperti di dominio inseriscono annotazioni minimali al codice sorgente per esprimere l'intenzione di eseguire il programma 
su pi\`u processori simultaneamente.
