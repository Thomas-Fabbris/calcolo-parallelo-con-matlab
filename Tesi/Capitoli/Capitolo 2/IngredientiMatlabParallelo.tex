%!TeX root = ../../Tesi.tex
\nocite{Sharma2009}
\subsection{Una breve prospettiva storica}
L'approccio seguito dai progettisti di The MathWorks\footnote{La \textit{software house} statunitense, con sede in Massachusetts (Stati Uniti), che si occupa dello sviluppo di MATLAB e di altri prodotti per il calcolo scientifico.}
per estendere MATLAB  al mondo del calcolo parallelo \`e stato modificare le caratteristiche del linguaggio 
stesso, cominciando dall'aggiunta di \textit{routine} comunemente impiegate nella risoluzione di problemi \textit{embarrassingly parallel}. 

A partire dai primi passi compiuti in questa direzione negli anni Ottanta del secolo scorso da Cleve Moler, l'autore del linguaggio, ci si imbatt\'e
nel fatto che il modello di memoria globale di MATLAB, secondo il quale le variabili definite dall'utente e importate dall'esterno vengono conservate 
in un'area di memoria allocata dalla sessione attiva di MATLAB, era in contrasto con il modello di memoria condivisa impiegato dalla maggioranza dei sistemi 
multiprocessore.

Questa incompatibilit\`a causò dei rallentamenti al progetto di parallelizzazione di MATLAB, ma le pressioni esterne di coloro che auspicavano a un suo completamento erano troppo insistenti per essere ignorate.\newline
La crescente disponibilit\`a di sistemi multiprocessore aveva reso il calcolo parallelo un argomento presente sulla bocca di tutti gli specialisti del 
settore; la comparsa delle prime architetture \textit{multicore} e l'ascesa dei \textit{cluster} Beowulf\footnote{I \textit{cluster} Beowulf sono \textit{cluster} costituiti dall'interconessione di componenti hardware commerciali, ad esempio PC al termine della loro vita utile, mediante una tradizionale rete LAN. } permisero una diffusione massiccia dei sistemi di calcolo ad alte prestazioni, che precedette la \enquote{democratizzazione} dei WSC portata dal \textit{cloud computing}.\newline   
Inoltre, MATLAB era gi\`a allora un ambiente di programmazione affermato all'interno della comunit\`a scientifica e quindi doveva fornire ai propri utenti un prodotto completo e funzionale 
in tutti gli scenari applicativi, inclusi i progetti a elevata intensit\`a computazionale.

Ecco che nel novembre del 2004 vennero rilasciati al pubblico i primi risultati di questo progetto, sotto le vesti di due pacchetti software addizionali (chiamati \textit{toolbox} nel vocabolario tecnico del linguaggio): il Distributed Computing 
Toolbox\textsuperscript{\texttrademark} e il MATLAB Distributed Computing Engine\textsuperscript{\texttrademark}\footnote{I due nomi commerciali sono i corrispettivi degli odierni Parallel 
Computing Toolbox\textsuperscript{\texttrademark} e MATLAB Parallel Server\textsuperscript{\texttrademark}.}.

\subsection{Gli aspetti imprescindibili dell'implementazione}
\label{sec2.1.2}
L'adattamento di MATLAB al calcolo parallelo non fu condotto in modo casuale, bens\'i le aggiunte al linguaggio furono ponderate attentamente a partire dalle informazioni ricavate dai sondaggi condotti nelle fasi preliminari del progetto.\newline
Per questo motivo, il modello di programmazione proposto da MATLAB \`e adatto all'esecuzione di programmi paralleli su sistemi \textit{multicore} e su \textit{cluster} di elaboratori, trattandosi delle architetture di calcolo parallelo pi\`u comuni in ambito industriale.

Di seguito elenchiamo, in ordine decrescente di importanza, gli obiettivi di progettazione che ispirarono e continuano a ispirare il processo di parallelizzazione di MATLAB;
\begin{itemize}
    \item la programmabilit\`a, cio\'e la capacit\`a di creare programmi che soddisfino i requisiti degli utenti e che siano facili da mantenere per gli sviluppatori;
    \item l'esecuzione di codice arbitrario sui sistemi multiprocessore supportati;
    \item l'astrazione da dettagli irrilevanti durante l'implementazione; di conseguenza, lo sviluppatore medio non deve pi\`u preoccuparsi di aspetti quali lo \textit{scheduling} delle \textit{task} e la distribuzione dei dati alle unit\`a di lavoro, in quanto vengono gestiti automaticamente dal linguaggio;
    \item l'indipendenza del programma dall'allocazione delle risorse computazionali: un software parallelo scritto in MATLAB deve funzionare correttamente sia quando eseguito su un sistema multiprocessore che su un sistema monoprocessore, adattando il suo comportamento alle risorse di calcolo disponibili;
    \item l'accesso a costrutti di programmazione di prima classe\footnote{Secondo la classificazione proposta dall'informatico britannico Christopher Strachey \cite{SICP96}, i costrutti di programmazione di prima classe possono essere manipolati liberamente nelle istruzioni del linguaggio; in pratica, devono poter essere passati come parametri attuali durante l'invocazione di una procedura, restituiti come valore di ritorno di una funzione e assegnati a variabili o a strutture dati.}. 
\end{itemize}
Il percorso di trasformazione di MATLAB al fine di renderlo appetibile al calcolo parallelo non \`e ancora giunto al termine. \newline La destinazione finale fissata 
dagli addetti ai lavori \`e la realizzazione del modello di linguaggio ideato dal direttore tecnico di TheMathWorks, Roy Lurie \cite{Lurie2007},
secondo cui gli esperti di dominio inseriscono annotazioni minimali al codice sorgente per esprimere l'intenzione di eseguire il programma 
su pi\`u processori simultaneamente.
