%!TeX root = ../../Tesi.tex
Il Parallel Computing Toolbox permette di risolvere problemi \textit{compute-intesive} e \textit{data-intensive} sfruttando
la capacit\`a di calcolo offerta dai pi\`u recenti microprocessori \textit{multicore} e \textit{cluster} di elaboratori. \newline
Costrutti di programmazione di alto livello, come gli \textit{array} distribuiti, consentono di sviluppare applicazioni MATLAB scalabili senza ricorrere alla programmazione
MPI\footnote{La \textit{Message Passing Interface} (MPI) rappresenta lo standard per il modello di comunicazione interprocesso basato sullo scambio
    di messaggi nei sistemi distribuiti\,\cite{NMSU2025}, stabilendo la sintassi e la semantica di funzioni di libreria impiegate nella scrittura di software parallelo in C, \CC e Fortran.}.\newline
L'applicazione pu\`o essere eseguita su \textit{cluster} o su server in \textit{cloud} senza dover apportare alcuna modifica al codice grazie a MATLAB
Parallel Server, cos\`i da concentrarsi esclusivamente sullo sviluppo dell'algoritmo migliore per il caso d'uso in esame.
\subsection{L'architettura e i componenti chiave}
Incominciamo il nostro studio del Parallel Computing Toolbox riportando alcune definizioni, tratte dalla documentazione ufficiale di MATLAB\,\cite{TheMathworksIncWhatParallelComputing}, circa gli strumenti messi a disposizione dal \textit{toolbox}.
\begin{itemize}
    \item \textit{Client}: termine impiegato per identificare la sessione con cui l'utente sta interagendo; tipicamente coincide con il
          computer usato dallo sviluppatore durante la prototipazione del programma a esecuzione parallela.\newline
          Attraverso le funzioni che compongono il Parallel Computing Toolbox, il \textit{client} suddivide la computazione in \textit{task} atomiche e le assegna ai \textit{worker}.
    \item \textit{Parallel Pool}: spesso abbreviato in parpool, \`e definito come un insieme di \textit{worker} comunicanti che possono eseguire codice interattivamente.
    \item \textit{Worker}: corrisponde a un'istanza di MATLAB priva di interfaccia grafica, in grado di fornire la potenza del
          motore di calcolo del linguaggio.
\end{itemize}

Una prima distinzione da rimarcare \`e quella esistente fra l'infrastruttura e i componenti del linguaggio inclusi negli strumenti di calcolo parallelo di MATLAB.\newline
Il linguaggio comprende costrutti di programmazione parallela e funzioni con supporto automatico al parallelismo, mentre l'infrastruttura riguarda i meccanismi che coaudivano il linguaggio, come il protocollo di trasferimento del codice e dei dati alle unit\`a di lavoro.

L'architettura del \textit{cluster} di elaboratori impiegata per l'analisi del Parallel Computing Toolbox \`e schematizzata in figura \ref{fig:architetturaRiferimento}.\newline
MATLAB Parallel Server riunisce un insieme di \textit{worker}, in esecuzione sui nodi del \textit{cluster}, che ricevono le
\textit{task} computazionali assegnate dal \textit{client} attraverso specifiche funzioni del Parallel Computing Toolbox. \newline
I \textit{worker} leggono il codice da eseguire e i dati su cui lavorare da una memoria di massa condivisa popolata dall'\textit{head node}
(non rappresentato in figura), un nodo del sistema responsabile della schedulazione delle attivit\`a.\newline
Una volta terminata l'elaborazione, i risultati sono raccolti dall'\textit{head node} e posti all'interno dello spazio di lavoro del \textit{client}
mediante i canali di comunicazione instaurati tra quest'ultimo e i \textit{worker}.

\begin{figure}[htbp]
    \centering
    \includegraphics[width=0.8\textwidth]{../Risorse/Capitolo 2/ReferenceArchitecture.png}
    \caption{Architettura di riferimento per gli strumenti di calcolo parallelo di MATLAB
        \small{\textit{(Da} \url{https://it.mathworks.com/products/MATLAB-parallel-server.html})}.}
    \label{fig:architetturaRiferimento}
\end{figure}

Giunti a questo punto, introduciamo il paradigma di programmazione parallela di MATLAB e le funzionalit\`a del motore di calcolo del linguaggio dedicate al calcolo parallelo
\begin{itemize}
    \item parallelizzazione implicita: alcune funzioni, quando richiamate dal codice sorgente del programma, sfruttano le librerie di \textit{runtime} del linguaggio
          in modo da essere eseguite su \textit{thread} distinti all'interno della stessa sessione;
    \item parallelizzazione esplicita: il carico di lavoro del programma \`e automaticamente suddiviso in \textit{task} elementari, ciascuna delle quali viene
          attribuita a un \textit{worker} per l'esecuzione.
\end{itemize}

Nelle prossime sezioni, esamineremo da vicino alcuni costrutti paralleli propri del linguaggio, astraendo dall'infrastruttura sottostante, malgrado entrambe le componenti
siano imprescindibili per il corretto funzionamento del Parallel Computing Toolbox.

\subsection{La parallelizzazione implicita e il \textit{multithreading} nativo}
I \textit{toolbox} di MATLAB sono dotati di un crescente numero di funzioni con supporto automatico al parallelismo, al fine di beneficiare di tutti
i vantaggi dati dall'elaborazione parallela, senza modificare il codice eventualmente scritto per la versione seriale di un programma, in accordo con i principi di design elencati
nella sezione \ref{sec:aspettiImplementazioneMATLABParallelo}.

Alcune funzioni, come \lstinline{mldivide} per la risoluzione di sistemi di equazioni lineari, sono eseguite automaticamente su \textit{thread}
distinti, se richiamate dalla sessione attiva di MATLAB.

Ragionando sulla nostra architettura di riferimento, la parallelizzazione implicita viene attivata solo quando la funzione \`e eseguita direttamente dal \textit{client},
mentre \`e sconsigliata nei casi in cui l'esecuzione \`e a carico dei nodi del \textit{cluster}, allo scopo di evitare un parallelismo \enquote{annidato} che degraderebbe le prestazioni
dell'intero sistema. \newline
In quest'ottica, possiamo notare come i progettisti del linguaggio abbiano ideato i \textit{worker} come delle unit\`a di elaborazione a singolo \textit{thread}.

Quando il \textit{client} incontra una funzione con supporto automatico al parallelismo nel codice sorgente del programma, avvia un \textit{parallel pool} per la sua esecuzione in parallelo. \newline
Un apposito profilo di configurazione determina le caratteristiche dell'ambiente di elaborazione parallela; nello specifico, il Parallel Computing Toolbox permette di scegliere tra i
seguenti profili preimpostati:
\begin{itemize}
    \item \textit{Processes}: i \textit{worker} vengono attivati come processi in esecuzione sui \textit{core} fisici del calcolatore che ospita la sessione
          principale di MATLAB.
    \item \textit{Threads}: i \textit{worker} sono in esecuzione su dei \textit{thread} e non pi\`u su processi veri e propri.
\end{itemize}
I vantaggi portati dall'ambiente
parallelo \textit{Threads} sono un minor consumo di memoria, un basso costo di comunicazione tra i \textit{worker} e una schedulazione delle attivit\`a particolarmente
performante, a scapito della disponibilit\`a di un'ampia gamma di funzioni con supporto alla parallelizzazione implicita su \textit{thread}.

Relativamente alla scelta del numero di \textit{worker} nell'ambiente \textit{Processes}, \'e consigliato riservare un motore di calcolo per ogni \textit{core}
fisico disponibile, ignorando la presenza di eventuali \textit{core} virtuali; infatti, questi ultimi condividono alcune risorse di calcolo appartenenti allo
stesso processore, tra cui la FPU (\textit{Floating Point Unit}), e poich\'e la quasi totalit\`a delle elaborazioni in MATLAB richiede l'esecuzione di operazioni
aritmetiche in virgola mobile, limitare il numero di \textit{worker} per CPU a uno migliora la stabilit\`a del sistema. \newline
L'unica eccezione \`e rappresentata dalle applicazioni \textit{data-intensive}, per le quali potrebbe essere conveniente innalzare il numero di \textit{worker} per
\textit{core} a due.

Indipendentemente dall'ambiente di esecuzione selezionato, un singolo parpool a supporto della parallelizzazione implicita pu\`o contenere al pi\`u 512 \textit{worker}, a prescindere dalle specifiche
del calcolatore utilizzato.
\begin{figure}[htbp]
    \centering
    \includegraphics[width=0.8\textwidth]{../Risorse/Capitolo 2/ImplicitParallelization.png}
    \caption{Rappresentazione del modello di parallelizzazione implicita di MATLAB su un sistema \textit{dual-core}
        \small{\textit{(Da} \url{https://it.mathworks.com/discovery/MATLAB-multicore.html})}.}
    \label{fig:parallelismoImplicito}
\end{figure}\newline
Se una funzione non include il supporto automatico al parallelimo, possiamo trasferire l'esecuzione del programma su una \textit{workstation}, in modo da beneficiare
dello \textit{speedup} garantito da un sistema con maggiore capacit\`a di calcolo, oppure possiamo utilizzare il paradigma di programmazione parallela esplicita offerto
da MATLAB.

\subsection{Il paradigma di programmazione parallela esplicita}

Il modello di programmazione parallela esplicita esposto dal Parallel Computing Toolbox \`e un'applicazione del parallelismo a livello di dati che si fonda sull'esistenza di costrutti di programmazione
parallela di cui i programmatori possono avvalersi durante il processo di sviluppo.

Il meccanismo sfruttato a bassissimo livello per la realizzazione di programmi a elaborazione parallela \`e la comunicazione basata su scambio di messaggi tra \textit{worker}
appertenenti a un medesimo \textit{parallel pool}, ma questo approccio viene spesso criticato, essendo considerato l'equivalente del linguaggio
Assembly per il calcolo parallelo.\newline
Con l'obiettivo di agevolare la scrittura di software parallelo, alcuni costrutti di programmazione di alto livello sono stati introdotti nel linguaggio, aumentando il livello di
astrazione del codice e consentendo la stesura di algoritmi paralleli simili alle loro controparti seriali.

Un esempio di costrutto di programmazione parallela \`e incarnato dagli \textit{array}\footnote{Secondo la terminologia di MATLAB, la parola \textit{array} \`e un termine universale per riferirsi a strutture dati ospitanti vettori riga, vettori colonna o matrici.}
distribuiti, strutture dati il cui contenuto viene partizionato tra i \textit{worker} di uno stesso \textit{cluster}. Ogni \textit{worker} conserva in locale una porzione dell'\textit{array} distribuito, ma all'occorrenza pu\`o comunicare con gli altri \textit{worker} per accedere a tutti gli elementi dell'\textit{array}.\newline
L'impiego degli \textit{array} distribuiti permette di memorizzare strutture dati di dimensioni tali da non poter essere contenute nella memoria centrale di un
singolo calcolatore, sfruttando la capacit\`a di memorizzazione offerta dalla combinazione di tutti i nodi del \textit{cluster}.

Gli \textit{array} distribuiti possono contenere dati di qualsiasi tipo e supportano la distribuzione degli elementi lungo una dimensione,
per riga oppure per colonna.\newline
A questo proposito, dobbiamo precisare che \`e prevista la possibilit\`a di definire distribuzioni dei dati alternative e che, molto spesso, il partizionamento degli elementi viene manipolato implicitamente dall'esecuzione di certe operazioni
come \lstinline{gather}, una funzione utile a riunire un \textit{array} distribuito all'interno dello spazio di lavoro del \textit{client}.

Un ulteriore vantaggio derivante dall'uso degli \textit{array} distribuiti \`e l'assenza di differenze sintattiche nell'accesso agli elementi rispetto agli \textit{array} tradizionali; l'infrastruttura sottostante garantisce, in ogni momento, una distribuzione dei dati idonea all'esecuzione delle operazioni
richieste dall'utente.\newline
Questo ampio raggio di manovra lasciato al programmatore potrebbe introdurre consistenti \textit{overhead} di comunicazione tra i \textit{worker},
ma la programmabilit\`a rappresenta l'obiettivo di design prioritario nel processo di parallelizzazione di MATLAB, surclassando persino le prestazioni.

Centinaia di funzioni native, e altrettante presenti nei \textit{toolbox} sviluppati dalla \textit{community} di utenti, sono state progettate per lavorare sinergicamente con gli \textit{array} distribuiti.\newline
Ad esempio, MATLAB prevede un esteso insieme di funzioni parallele che operano su \textit{array} distribuiti, implementate a partire dalle procedure definite nella libreria di algebra lineare numerica ScaLAPACK (\textit{Scalable Linear Algebra Package}).