I primi progettisti di calcolatori, negli anni Cinquanta del Novecento, ebbero l'intuizione di interconnettere
una moltitudine di calcolatori tradizionali, al fine di ottenere un sistema di elaborazione sempre più potente.\newline
Quel sogno primordiale port\`o alla nascita dei \textit{cluster} di elaboratori trent'anni dopo e allo sviluppo delle architetture di microprocessore
\textit{multicore} a partire dall'inizio del 2000.\newline
Oggi la maggior parte delle applicazioni in ambito scientifico, tra cui quelle impiegate nella risoluzione di problemi di analisi numerica
su larga scala, possono funzionare solo disponendo di sistemi di calcolo in grado di fornire una capacit\`a di elaborazione molto elevata.

Nel capitolo \ref{cap1} ci concentreremo sul concetto di calcolo parallelo e sulle principali sfide da affrontare
durante la scrittura di software eseguito su pi\`u processori simultaneamente, tra cui spicca una crescita delle prestazioni non proporzionale
al miglioramento apportato al sistema di elaborazione, un risultato espresso quantitativamente dalla legge di Ahmdal.

Nel corso del capitolo \ref{cap2}, analizzeremo i principali costrutti di programmazione parallela messi a disposizione dall’ambiente di calcolo numerico
e programmazione MATLAB\textsuperscript{\textregistered}, nonch\'e le scelte di progettazione fondamentali che hanno influenzato
le attuali caratteristiche del linguaggio dedicate alla scrittura di programmi a esecuzione parallela.

Nel capitolo 3 forniremo un'illustrazione formale del metodo di Jacobi, un metodo iterativo dell’analisi numerica per la risoluzione
approssimata di sistemi di equazioni lineari, dopo aver introdotto alcune nozioni di algebra lineare la cui conoscenza \`e necessaria per un adeguato
sviluppo dell'argomento.\newline
Successivamente, proporremo un’implementazione parallela dell'algoritmo codificato dal metodo di Jacobi, sfruttando le potenzialità fornite
dall'impiego dagli \textit{array} globali per aumentare il livello di astrazione del programma a elaborazione parallela.\newline
Infine, ci occuperemo dell’analisi dei risultati ottenuti dall’esecuzione dell’algoritmo su problemi di grandi dimensioni.
