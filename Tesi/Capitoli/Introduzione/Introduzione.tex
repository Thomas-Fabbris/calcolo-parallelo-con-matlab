I primi progettisti di calcolatori, negli anni '50 del secolo scorso, hanno avuto sin da subito l'intuizione di interconnettere
una moltitudine di calcolatori tradizionali, al fine di ottenere un sistema di elaborazione sempre più potente.
Questo sogno primordiale ha portato alla nascita dei \textit{cluster} di elaboratori trent'anni dopo e allo sviluppo delle architetture
\textit{multicore} a partire dall'inizio del 2000.
Oggi la maggior parte delle applicazioni in ambito scientifico, tra cui quelle impiegate nella risoluzione di problemi di analisi numerica
su larga scala, possono funzionare solo disponendo di sistemi di calcolo in grado di fornire una capacità di elaborazione molto elevata.

\vspace*{0.5cm}
Nel capitolo \ref{cap1} di questo lavoro di tesi, ci concentreremo sul concetto di calcolo parallelo e sulle principali sfide da affrontare
durante la scrittura di software eseguito su più processori simultaneamente, tra cui spicca una crescita delle prestazioni non proporzionale
al miglioramento apportato al sistema di elaborazione, un risultato espresso quantitativamente dalla legge di Ahmdal.

\vspace*{0.5cm}
Nel corso del capitolo 2, analizzeremo i principali costrutti di programmazione parallela messi a disposizione dall’ambiente di calcolo numerico
e programmazione MATLAB\textsuperscript{\textregistered}, nonché le scelte di progettazione fondamentali che hanno influenzato
le attuali caratteristiche del linguaggio dedicate alla scrittura di programmi ad esecuzione parallela.

\vspace*{0.5cm}
Nel capitolo 3, descriveremo dal punto di vista formale il metodo di Jacobi, un metodo iterativo dell’analisi numerica per la risoluzione
approssimata di sistemi di equazioni lineari, dopo aver introdotto alcune nozioni di algebra lineare la cui conoscenza è necessaria per un'adeguato
sviluppo dell'argomento.

Successivamente, proporremo un’implementazione parallela dell'algoritmo dietro al metodo di Jacobi, sfruttando le potenzialità fornite
dagli \textit{array} globali, utili per aumentare il livello di astrazione di un programma parallelo.

Infine, ci occuperemo dell’analisi dei risultati ottenuti dall’esecuzione dell’algoritmo su problemi di grandi dimensioni.
