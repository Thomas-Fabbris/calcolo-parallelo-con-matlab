%!TeX root = ../../Tesi.tex
I primi progettisti di calcolatori, negli anni Cinquanta del Novecento, ebbero l'intuizione di interconnettere
una moltitudine di calcolatori tradizionali, al fine di ottenere un sistema di elaborazione sempre più potente.
Quel sogno primordiale port\`o alla nascita dei \textit{cluster} di elaboratori trent'anni dopo e allo sviluppo delle architetture per microprocessori
\textit{multicore} a partire dall'inizio del 2000.\newline
Oggi, la maggior parte delle applicazioni in ambito scientifico, tra cui quelle impiegate nella simulazione di processi fisici e chimici a livello molecolare, possono funzionare solo disponendo di sistemi di calcolo caratterizzati da una capacit\`a di elaborazione molto elevata.

Nel capitolo \ref{cap:calcoloParalleloSfidaOpportunita} ci soffermeremo sul concetto di calcolo parallelo e sulle principali sfide da affrontare
durante la scrittura di software eseguito su pi\`u processori simultaneamente, tra cui spicca una crescita delle prestazioni non proporzionale
al miglioramento apportato al sistema di elaborazione.

Nel corso del capitolo \ref{cap:unLinguaggioPerIlCalcoloParalleloMATLAB}, analizzeremo i principali costrutti di programmazione parallela messi a disposizione dall’ambiente di calcolo numerico
e programmazione MATLAB\textsuperscript{\textregistered}\footnote{MATLAB \`e un marchio registrato da The MathWorks, Inc. Per un elenco di ulteriori marchi, visitare la pagina \url{https://www.mathworks.com/trademarks}.}nonch\'e le scelte di progettazione fondamentali che hanno influenzato
le attuali caratteristiche del linguaggio dedicate alla scrittura di programmi a esecuzione parallela.

Nel capitolo \ref{cap:metodoJacobiParallelo} forniremo un'illustrazione formale del metodo di Jacobi, un metodo iterativo dell’analisi numerica per la risoluzione
approssimata di sistemi di equazioni lineari. Successivamente, proporremo un’implementazione dell'algoritmo codificato dal metodo iterativo, sfruttando le potenzialità fornite
dagli \textit{array} distribuiti. In ultima istanza, ci occuperemo dell’analisi prestazionale di un problema di esempio, fondamentale per supportare i concetti teorici spiegati nei capitoli precedenti.
