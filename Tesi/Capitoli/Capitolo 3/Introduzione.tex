%!TeX root = ../../Tesi.tex
Innumerevoli modelli matematici, come il modello di reazione-diffusione in chimica, si formulano per mezzo di equazioni
differenziali. \newline
Non \`e una novit\`a il fatto che, per la maggior parte delle equazioni differenziali, l'integrale generale non sia esprimibile in forma esplicita, motivo per il quale \`e
necessario ricorrere a metodi di integrazione numerica per la loro risoluzione.\newline
Tra questi, spicca il metodo degli elementi finiti, un metodo per l'approssimazione numerica di equazioni differenziali
attraverso sistemi di equazioni lineari.

In aggiunta, i problemi realmente affrontati dalle scienze applicate dipendono da migliaia di variabili, per cui lo sviluppo di metodi efficienti per la risoluzione di sistemi lineari di grandi
dimensioni non \`e un capriccio squisitamente teorico, ma trova fondamentali applicazioni in tutti i settori in cui la modellistica viene impiegata.

L'obiettivo principale di questo capitolo \`e la presentazione del metodo di Jacobi, un metodo iterativo per la risoluzione di sistemi di equazioni lineari.\newline
I concetti esposti nei capitoli \ref{cap:calcoloParalleloSfidaOpportunita} e \ref{cap:unLinguaggioPerIlCalcoloParalleloMATLAB} torneranno utili nell'affrontare questo problema, matematicamente elementare ma computazionalmente
intrigante, sfruttando l'elevata capacit\`a di calcolo messa a disposizione dai sistemi a elaborazione parallela.\newline
Nei paragrafi da \ref{par:sistemiLineari} a \ref{par:metodoJacobi}, faremo riferimento a svariati risultati propri dell'algebra lineare e dell'analisi numerica, indispensabili per una trattazione accurata degli
argomenti in questione; rimandandiamo a \cite{Betti2000} e a \cite{Quarteroni2002} per le relative dimostrazioni.

Per concludere proporremo un esempio di applicazione del metodo di Jacobi nel paragrafo \ref{par:applicazioneMetodoJacobi},
particolarmente significativo per illustrare in pratica quanto spiegato nei capitoli precedenti relativamente alla natura del calcolo parallelo
e alle caratteristiche dell'ambiente MATLAB a esso dedicate.