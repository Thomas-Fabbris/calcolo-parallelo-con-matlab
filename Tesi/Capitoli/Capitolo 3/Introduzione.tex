%!TeX root = ../../Tesi.tex
Innumerevoli modelli matematici, come il modello reazione-diffusione in chimica, si formulano per mezzo di equazioni 
differenziali. \newline
Non \`e una novit\`a il fatto che, per la maggior parte delle equazioni differenziali, l'integrale generale non sia esprimibile in forma esplicita, motivo per il quale \`e 
necessario ricorrere a metodi di integrazione numerica per la loro risoluzione.\newline
Tra questi, spicca il metodo degli elementi finiti, un metodo per l'approssimazione di equazioni differenziali 
attraverso sistemi di equazioni lineari.

I problemi reali, in campo ingegneristico e fisico, dipendono da migliaia di variabili, per cui lo sviluppo di metodi efficienti per la risoluzione sistemi lineari di grandi 
dimensioni non \`e un capriccio squisitamente teorico, ma trova fondamentali applicazioni in tutti i settori in cui la modellistica viene impiegata.

L'obiettivo di questo capitolo \`e la presentazione del metodo di Jacobi, un metodo iterativo per la risoluzione di sistemi di equazioni lineari, e la sua 
applicazione a problemi di grandi dimensioni.\newline
I concetti esposti nei capitoli \ref{cap1} e \ref{cap2} torneranno utili nell'affrontare questo problema, matematicamente elementare ma computazionalmente 
intrigante, sfruttando l'elevata capacit\`a di calcolo messa a disposizione dai sistemi a elaborazione parallela.

Di seguito, faremo riferimento a svariati risultati propri dell'algebra lineare e dell'analisi numerica, indispensabili per una trattazione accurata degli 
argomenti in questione; rimandiamo a \cite{Betti2000} e a \cite{Quarteroni2000} per le relative dimostrazioni.