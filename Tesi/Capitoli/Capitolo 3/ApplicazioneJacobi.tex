%!TeX root = ../../Tesi.tex
\nocite{MathWorksIterativeMethods}
Ora presentiamo un esempio di risoluzione di un sistema di equazioni lineari del tipo $A\mathbf{x}=\mathbf{x}$ mediante la funzione 
\lstinline{jacobi} discussa nel paragrafo \ref{par:algoritmoJacobi}.

Consideriamo come matrice dei coefficienti $A$ la matrice derivante dalla discretizzazione dell'equazione di Poisson su un 
dominio quadrato $\Omega=[0, 1]\times[0, 1]$ mediante il metodo alle differenze con 5 punti su una griglia bidimensionale $\mathcal{G}$ di dimensione $n\times n$\footnote{
Nonostante l'equazione di Poisson sia un'equazione differenziale alle derivate parziali di fondamentale importanza in meccanica, 
elettrostatica e fisica tecnica, non \`e nostra intenzione descrivere il problema dal punto di vista strettamente matematico.}.\newline
Questa matrice viene generata attraverso i comandi MATLAB
\begin{matlabcode}
clc
clear
n = 10;
N = n^2;
A = gallery('poisson', n);
fprintf(['\nDimensione matrice dei coefficienti A: '...
       '%u x %u \n\n'],N,N);
\end{matlabcode}
\begin{matlaboutput}
Dimensione matrice dei coefficienti A: 100 x 100 
\end{matlaboutput}
\begin{matlabcode}
disp(full(A(1:5, 1:5)));
\end{matlabcode}
\begin{matlaboutput}
     4    -1     0     0     0
    -1     4    -1     0     0
     0    -1     4    -1     0
     0     0    -1     4    -1
     0     0     0    -1     4
\end{matlaboutput}
Notiamo che $A\in\mathbb{R}^{n^{2} \times n^{2}}$ per cui il sistema lineare da risolvere si presenta indubbiamente 
come un problema di grandi dimensioni.\newline
Inoltre, $A$ \`e, per costruzione, una matrice quadrata a dominanza diagonale stretta per righe, il che la rende una candidata 
ideale per l'applicazione del metodo di Jacobi alla risoluzione del sistema di equazioni da essa descritto.

Scegliamo come vettore dei termini noti $\mathbf{b}\in\mathbb{R}^{{n}^{2}}$ 
\begin{equation*}
\mathbf{b} = \sum_ {A_{i} \in J} A_{i}
\end{equation*}
dove $J = \{A_{1}, A_{2}, ..., A_{N}\}$ denota l'insieme delle colonne della matrice $A$.
\begin{matlabcode}
b = sum(A,2);
\end{matlabcode}
Possiamo dimostrare che la soluzione $\mathbf{x}$ del sistema sia pari a 
\begin{equation*}
\mathbf{x} = [1, 1, ..., 1]^\top
\end{equation*}
