Nell'appendice A forniamo un'implementazione del metodo di Jacobi sotto forma di una funzione MATLAB che trae vantaggio delle funzionalit\`a dedicate al 
calcolo parallelo, tra cui gli \textit{array} distribuiti e il parallelismo implicito a livello di \textit{thread}, gi\`a presentate nel capitolo 
\ref{cap2}.

L'interfaccia della funzione \lstinline|jacobi|, cos\`i come il codice vero e proprio, sono stati ispirati dalla funzione MATLAB nativa 
\lstinline|pcg| per la risoluzione di sistemi di equazioni lineari mediante il metodo del gradiente coniugato precondizionato \, \cite{MathWorksPCG} 
(\textit{Preconditioned Conjugate Gradients method}), un metodo iterativo spesso considerato come alternativa al metodo di Jacobi per la risoluzione 
di alcune classi di sistemi per via delle sue notevoli propriet\`a di convergenza.\newline
Nello specifico, il prototipo della funzione MATLAB sviluppata \`e
\begin{lstlisting}
    [x,flag,relres,iter,resvec]=jacobi(A,b,tol,maxit,x0)~\text{.}~
\end{lstlisting}

La funzione \lstinline|jacobi| si aspetta di ricevere in ingresso i seguenti parametri obbligatori:
\begin{itemize}
    \item \lstinline|A|, la matrice dei coefficienti $A$ del sistema \eqref{eq:formaMatricialeSistemiLineari};
    \item \lstinline|b|, il vettore colonna dei termini noti $\mathbf{b}$ del sistema;
\end{itemize}
a cui si aggiungono alcuni parametri facoltativi:
\begin{itemize}
    \item \lstinline|tol|, la tolleranza $\varepsilon$ nell'approssimazione della soluzione del sistema;
    \item \lstinline|maxit|, il numero massimo di iterazioni consentite prima di giungere a una soluzione approssimata;
    \item \lstinline|x0|, il vettore iniziale $\mathbf{x}^{(0)}$ della successione $\mathbf{\{x^{(k)}\}}$ costruita dal metodo iterativo.
\end{itemize}

Gli argomenti restituiti in output da \lstinline|jacobi| sono:
\begin{itemize}
    \item \lstinline|x|, la soluzione $\mathbf{x}$ del sistema in questione;
    \item \lstinline|flag|, un valore numerico indicante lo stato di uscita dall'esecuzione della funzione. La tabella riassume i possibili valori di \lstinline|flag| e il risultato di convergenza a essi associato;
    \item \lstinline|iter|, il numero dell'iterazione a cui la soluzione \lstinline|x| del sistema \`e stata calcolata;
    \item \lstinline|resvec|, un \textit{array} in cui ciascun elemento rappresenta il residuo del sistema a ogni passo dell'algoritmo;
    \item \lstinline|relres|, il residuo normalizzato ottenuto al termine dell'esecuzione del metodo di Jacobi.
\end{itemize}
\begin{table}[htbp]
    \centering
    \begin{tabular}{c p{0.75\textwidth}}
        \hline
        \textbf{Flag} & \textbf{Risultato di convergenza} \\
        \hline
        \\[-1..5em]
        0 & \lstinline|jacobi| \`e riuscito a convergere alla soluzione \lstinline|x|, secondo la tolleranza desiderata \lstinline{tol}, entro \lstinline{maxit} iterazioni. \\[1em]
        1 & \lstinline{jacobi} ha completato \lstinline{maxit} iterazioni, senza raggiungere la convergenza. \\[1em]
        2 & Una delle quantit\`a scalari determinate dall'algoritmo \`e diventata troppo piccola o troppo grande per proseguire con l'esecuzione. \\[1em]
        3 & \lstinline|jacobi| \`e caduto in stagnazione dopo che un certo numero di iterazioni consecutive hanno restituito gli stessi risultati. \\
        \hline
    \end{tabular}
    \caption{Valori assunti dal parametro di output \lstinline{flag} della funzione \lstinline{jacobi} e relativi risultati di convergenza.}
\end{table}