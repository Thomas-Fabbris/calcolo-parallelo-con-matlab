%!TeX root = ../../Tesi.tex
Nell'appendice \ref{app:jacobi} riportiamo un'implementazione del metodo di Jacobi che trae vantaggio delle funzionalit\`a del linguaggio MATLAB dedicate al 
calcolo parallelo presentate nel capitolo \ref{cap2}.

\subsection{Prototipo e semantica dei parametri}
L'interfaccia della funzione, cos\`i come il codice vero e proprio, sono ispirati dalla funzione MATLAB 
\lstinline|pcg| per la risoluzione di sistemi di equazioni lineari mediante il metodo del gradiente coniugato precondizionato\,\cite{MathWorksPCG} 
un metodo iterativo spesso considerato un'alternativa al metodo di Jacobi grazie alle sue propriet\`a di convergenza.

Nello specifico, il prototipo della funzione MATLAB \`e
\begin{lstlisting}
[x,flag,relres,iter,resvec]=jacobi(A,b,tol,maxit,x0)
\end{lstlisting}
La funzione \lstinline|jacobi| si aspetta di ricevere come parametri la matrice dei coefficienti $A$ del sistema \eqref{eq:formaMatricialeSistemiLineari} e il rispettivo vettore colonna dei termini noti $\mathbf{b}$.\newline
In aggiunta, l'utente pu\o` specificare i seguenti parametri opzionali:
\begin{itemize}
    \item \lstinline|tol|, la tolleranza $\varepsilon$ nell'approssimazione della soluzione del sistema;
    \item \lstinline|maxit|, il massimo numero di iterazioni consentite;
    \item \lstinline|x0|, il vettore iniziale della successione $\mathbf{\{x^{(k)}\}}$ costruita dal metodo iterativo.
\end{itemize}

Gli argomenti restituiti in output da \lstinline|jacobi| sono:
\begin{itemize}
    \item \lstinline|x|, la soluzione $\mathbf{x}$ del sistema in questione;
    \item \lstinline|flag|, un valore numerico indicante lo stato di uscita dall'esecuzione dell'algoritmo. La tabella \ref{tab:flagJacobi} riassume i possibili valori assunti da \lstinline|flag| e il risultato di convergenza a essi associato;
    \item \lstinline|iter|, il numero dell'iterazione a cui la soluzione \lstinline|x| del sistema \`e stata calcolata;
    \item \lstinline|resvec|, un \textit{array} in cui ciascun elemento rappresenta il residuo del sistema a ogni passo della risoluzione;
    \item \lstinline|relres|, il residuo normalizzato al termine dell'esecuzione del metodo.
\end{itemize}
\begin{table}[htbp]
    \renewcommand{\arraystretch}{1.2}
    \centering
    \begin{tabularx}{\textwidth}{@{} >{\centering\arraybackslash}m{1.5cm} X @{}}
        \toprule
        Flag & Risultato di convergenza \\
        \midrule
        0 & \lstinline{jacobi} è riuscito a convergere alla soluzione \lstinline{x}, secondo la tolleranza desiderata \lstinline{tol}, entro il numero massimo di iterazioni \lstinline{maxit}. \\
        \addlinespace
        1 & \lstinline{jacobi} ha raggiunto il numero massimo di iterazioni \lstinline{maxit} senza convergere alla tolleranza richiesta. \\
        \addlinespace
        2 & L'algoritmo si è interrotto poiché una delle quantità scalari calcolate è diventata troppo piccola o troppo grande per continuare l'esecuzione. \\
        \bottomrule
    \end{tabularx}
    \caption{Valori assunti dal parametro di output \lstinline{flag} della funzione \lstinline{jacobi} e relativi risultati di convergenza.}
    \label{tab:flagJacobi}
\end{table}
\subsection{Scelte progettuali}
La funzione \lstinline{jacobi} realizza il metodo di Jacobi per la risoluzione di sistemi lineari presentato nel paragrafo \ref{par:metodoJacobi} con l'unica differenza che l'algoritmo non agisce sui 
singoli elementi della matrice dei coefficienti, ma esegue le medesime operazioni su intere porzioni di dati.\newline
Questo approccio ci consente di beneficiare dell'\textit{overloading} degli operatori e delle \textit{routine} del linguaggio MATLAB, il cui comportamento viene adattato
alla classe di appartenenza degli operandi.\newline
A questo proposito, \lstinline{jacobi} pu\`o eseguire le operazioni aritmetiche in virgola mobile con una precisione singola oppure una precisione 
doppia, impostando di conseguenza la tolleranza predefinita.

Il criterio di arresto adottato \`e basato sul controllo del residuo normalizzato con l'introduzione di un'ulteriore condizione sul massimo numero di 
iterazioni che possono essere completate al fine di determinare una soluzione accettabile; mediamente, un valore di \lstinline{tol} minore richiede 
l'esecuzione di un maggior numero di passi prima di raggiungere una situazione di convergenza.

Il comportamento globale della funzione \`e fondato sull'ipotesi che l'utente fornisca dei parametri in input adeguati, ovvero che specifichi una matrice dei 
coefficienti non singolare e senza elementi diagonali nulli.\newline
Abbiamo deciso di introdurre un massimo numero di iterazioni possibili proprio per fermare l'esecuzione dell'algoritmo nel caso in cui il metodo venga 
applicato a sistemi con matrici malcondizionate o che non rispettino le ipotesi iniziali.

In accordo con i principi di \textit{design} del processo di parallelizzazione di MATLAB, esposti nel paragrafo \ref{par2.2}, il programma \`e 
indipendente dall'allocazione delle risorse computazionali: il codice sorgente contenuto nel file \lstinline{jacobi.m} pu\`o essere eseguito sia su un 
sistema monoprocessore che su un sistema multiprocessore senza notare differenze nelle soluzioni calcolate.

Inoltre, abbiamo sviluppato una versione specifica dello \textit{script} con supporto diretto agli \textit{array} distribuiti.\newline
Tramite un \textit{wrapper} della versione seriale di \lstinline{jacobi}, lo stesso algoritmo viene eseguito da un insieme di \textit{worker} in parallelo 
quando i parametri attuali sono di tipo distribuito.\newline
La scelta del particolare file sorgente da eseguire \`e stata delegata al \textit{dispatcher} del linguaggio, 
mentre tutti gli aspetti relativi alla computazione parallela, a partire dalla suddivisione del \textit{job} tra le unit\`a di lavoro, sono 
gestiti automaticamente dallo \textit{scheduler} integrato nell'infrastruttura del Parallel Computing Toolbox.

In ogni caso, abbiamo ridotto al minimo il numero di operazioni che necessitano il trasferimento di dati tra diversi \textit{worker}; ad esempio, 
il raccoglimento degli \textit{array} distribuiti viene forzato esclusivamente durante la fase finale di presentazione dei risultati.

