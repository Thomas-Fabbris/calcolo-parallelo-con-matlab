Nell'appendice A forniamo un'implementazione del metodo di Jacobi sotto forma di una funzione MATLAB che trae vantaggio delle funzionalit\`a dedicate al 
calcolo parallelo presentate nel capitolo \ref{cap2}.

L'interfaccia della funzione, cos\`i come il codice vero e proprio, sono stati ispirati dalla funzione MATLAB nativa 
\lstinline|pcg| per la risoluzione di sistemi di equazioni lineari mediante il metodo del gradiente coniugato precondizionato\,\cite{MathWorksPCG} 
(\textit{Preconditioned Conjugate Gradient method}), un metodo iterativo spesso visto come un'alternativa al metodo di Jacobi grazie alle sue propriet\`a di convergenza.\newline
Nello specifico, il prototipo della funzione MATLAB sviluppata \`e
\begin{lstlisting}
    [x,flag,relres,iter,resvec]=jacobi(A,b,tol,maxit,x0)
\end{lstlisting}

La funzione \lstinline|jacobi| si aspetta di ricevere in ingresso come parametri obbligatori la matrice dei coefficienti $A$ del sistema \eqref{eq:formaMatricialeSistemiLineari} e il vettore colonna dei termini noti $\mathbf{b}$.\newline
In aggiunta, possiamo esplicitare i seguenti parametri opzionali:
\begin{itemize}
    \item \lstinline|tol|, la tolleranza $\varepsilon$ nell'approssimazione della soluzione del sistema;
    \item \lstinline|maxit|, il numero massimo di iterazioni consentite;
    \item \lstinline|x0|, il vettore iniziale $\mathbf{x}^{(0)}$ della successione $\mathbf{\{x^{(k)}\}}$ costruita dal metodo iterativo.
\end{itemize}

Gli argomenti restituiti in output da \lstinline|jacobi| sono:
\begin{itemize}
    \item \lstinline|x|, la soluzione $\mathbf{x}$ del sistema in questione;
    \item \lstinline|flag|, un valore numerico indicante lo stato di uscita dall'esecuzione della funzione. La tabella \ref{tab:flagJacobi} riassume i possibili valori di \lstinline|flag| e il risultato di convergenza a essi associato;
    \item \lstinline|iter|, il numero dell'iterazione a cui la soluzione \lstinline|x| del sistema \`e stata calcolata;
    \item \lstinline|resvec|, un \textit{array} in cui ciascun elemento rappresenta il residuo del sistema a ogni passo dell'algoritmo;
    \item \lstinline|relres|, il residuo normalizzato al termine dell'esecuzione del metodo.
\end{itemize}

\begin{table}[htbp]
    \centering
    \begin{tabular}{c p{0.8\textwidth}}
        \hline
        \textbf{Flag} & \textbf{Risultato di convergenza} \\
        \hline
        \\[0.5em]
        0 & \lstinline|jacobi| \`e riuscito a convergere alla soluzione \lstinline|x|, secondo la tolleranza desiderata \lstinline{tol}, entro \lstinline{maxit} iterazioni. \\[1em]
        1 & \lstinline{jacobi} ha completato \lstinline{maxit} iterazioni, senza raggiungere la convergenza. \\[1em]
        2 & Una delle quantit\`a scalari determinate dall'algoritmo \`e diventata troppo piccola o troppo grande per proseguire con l'esecuzione. \\
        \\[0.5em]
        \hline
    \end{tabular}
    \caption{Valori assunti dal parametro di output \lstinline{flag} della funzione \lstinline{jacobi} e relativi risultati di convergenza.}
    \label{tab:flagJacobi}
\end{table}
\subsection{Scelte progettuali}
La funzione realizza il metodo di Jacobi presentato nel paragrafo \ref{par:metodoJacobi} con l'unica differenza che l'algoritmo non agisce sui 
singoli elementi della matrice \lstinline{A}, ma esegue le medesime operazioni su intere porzioni di dati.\newline
Questo approccio consente di sfruttare l'\textit{overloading} degli operatori del linguaggio MATLAB, il cui comportamento viene adattato in base 
alla classe di appartenenza degli operandi.\newline
A questo proposito, \lstinline{jacobi} pu\`o eseguire le operazioni aritmetiche in virgola mobile con una precisione singola oppure una precisione 
doppia, adattando di conseguenza la tolleranza predefinita dell'algoritmo.

Il criterio di arresto adottato \`e basato sul controllo del residuo normalizzato, con l'introduzione di un'ulteriore condizione sul massimo numero di 
iterazioni che la funzione pu\`o completare per raggiungere la tolleranza desiderata; mediamente, un valore di \lstinline{tol} minore richiede 
l'esecuzione di un maggior numero di passi prima di raggiungere una situazione di convergenza.

In accordo con i principi di progettazione seguiti nel processo di parallelizzazione di MATLAB, esposti nel paragrafo \ref{par2.2}, il programma \`e 
indipendente dall'allocazione delle risorse computazionali: il codice sorgente contenuto nel file \lstinline{jacobi.m} pu\`o essere eseguito sia su un 
sistema monoprocessore che su un sistema multiprocessore, senza alcuna differenza nei risultati ottenuti.

Inoltre, abbiamo sviluppato una versione specifica con supporto diretto agli \textit{array} distribuiti.\newline
Ricorrendo a un \textit{wrapper} della versione seriale di \lstinline{jacobi}, l'algoritmo viene eseguito da un insieme di \textit{worker} in parallelo 
quando i parametri attuali della funzione sono di tipo distribuito.\newline
La scelta dell'algoritmo da eseguire \`e stata delegata al \textit{dispatcher} del linguaggio, 
mentre tutti gli aspetti relativi all'esecuzione in parallelo della funzione, a partire dalla suddivisione del carico di lavoro tra le unit\`a di lavoro, sono 
gestiti automaticamente dallo \textit{scheduler} fornito dal Parallel Computing Toolbox.

In ogni caso, abbiamo ridotto al minimo il numero di operazioni che necessitano il trasferimento di dati tra diversi \textit{worker}; ad esempio, 
il raccoglimento degli \textit{array} distribuiti viene forzato esclusivamente durante la fase di visualizzazione dei risultati, al termine della computazione vera e propria.

