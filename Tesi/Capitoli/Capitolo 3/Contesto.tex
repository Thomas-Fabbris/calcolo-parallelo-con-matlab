\nocite{Quarteroni2000}
\nocite{Quarteroni1997}
Siano $A=(a_{ij})\in\mathbb{R}^{n\times n}$ una matrice quadrata di ordine $n\ge1$ a coefficienti reali e $\bm{b}=(b_{i})\in\mathbb{R}^{n}$ e $\bm{x}=(x_{i})\in\mathbb{R}^{n}$ dei vettori colonna di numeri reali.

Consideriamo il sistema di equazioni lineari in forma matriciale
\begin{equation}
\label{eq:formaMatricialeSistemiLineari}
A\bm{x}=\bm{b}
\end{equation}
,dove $A$ \`e la matrice dei coefficienti del sistema, $\bm{b}$ il vettore dei termini noti e $\bm{x}$ il vettore delle incognite.\newline
Possiamo notare come il sistema \eqref{eq:formaMatricialeSistemiLineari} corrisponda a un insieme di $n$ relazioni algebriche in $n$ incognite del tipo
\begin{equation}
\label{eq:formaAlgebricaSistemiLineari}
\sum_{j=1}^{n}a_{ij}x_{j}=b_{i},\quad i = 1, \dots, m
\end{equation}
,per il quale siamo interessati a determinare una o pi\`u soluzioni, ovvero trovare una qualsiasi $n$-upla di valori $x_{i}$ che soddisfi la \eqref{eq:formaMatricialeSistemiLineari}.
