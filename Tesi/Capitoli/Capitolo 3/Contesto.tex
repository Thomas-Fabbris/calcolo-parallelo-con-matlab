\nocite{Quarteroni2000}
\nocite{Quarteroni1997}
Siano $A=(a_{ij})\in\mathbb{R}^{n\times n}$ una matrice quadrata di ordine $n\ge1$ a coefficienti reali, $\bm{b}=(b_{i})\in\mathbb{R}^{n}$ 
e $\bm{x}=(x_{i})\in\mathbb{R}^{n}$ dei vettori colonna di numeri reali.

Consideriamo il sistema di equazioni lineari scritto in forma matriciale
\begin{equation}
\label{eq:formaMatricialeSistemiLineari}
A\bm{x}=\bm{b}
\end{equation}
dove $A$ \`e la matrice dei coefficienti del sistema, $\bm{b}$ il vettore dei termini noti e $\bm{x}$ il vettore delle incognite.\newline
Il sistema \eqref{eq:formaMatricialeSistemiLineari} rappresenta un insieme di $n$ relazioni algebriche in 
$n$ incognite del tipo
\begin{equation}
\label{eq:formaAlgebricaSistemiLineari}
\sum_{j=1}^{n}a_{ij}x_{j}=b_{i},\quad i = 1, \dots, n
\end{equation}
per il quale siamo interessati a determinarne le soluzioni, ovvero trovare delle $n$-uple di valori $x_{i}$ che 
soddisfino la \eqref{eq:formaAlgebricaSistemiLineari}.

Ricordiamo che l'esistenza e l'unicit\`a della soluzione di \eqref{eq:formaMatricialeSistemiLineari} \`e garantita se e solo se sono soddisfatte 
le seguenti condizioni, equivalenti tra di loro:
\begin{enumerate}
    \item $\det(A)\ne 0$, dove $\det(A)$ denota il determinante della matrice $A$;
    \item $A$ \`e invertibile;
    \item $\car(A)= n$, dove $\car(A)$ denota la caratteristica (o rango) di $A$, corrispondente al massimo numero di colonne (o righe) linearmente indipendenti della matrice;
    \item il sistema omogeneo associato $A\bm{x}=\bm{0}$ ammette come unica soluzione il vettore nullo.
\end{enumerate}

La soluzione del sistema $A\bm{x}=\bm{b}$ pu\`o essere espressa in forma chiusa tramite la regola di Cramer
\begin{equation}
    x_{j} = \frac{\Delta_{j}}{\det(A)},\quad j = 1, \dots, n
\end{equation}
con $\Delta_{j}$ il determinante della matrice ottenuta sostituendo la $j$-esima colonna di $A$ con il vettore dei termini noti $\bm{b}$.

Pur rappresentando un risultato fondamentale dell'algebra lineare, la regola di Cramer trova scarsa applicazione in ambito numerico per via del suo elevato 
costo computazionale.

Denotando con $\Delta_{ij}$ il determinante della matrice di ordine $n-1$ ottenuta da $A$ eliminando la $i$-esima riga e la $j$-esima colonna e con 
$A_{ij} = (-1)^{i+j}\Delta_{ij}$ il complemento algebrico dell'elemento $a_{ij}$, possiamo sfruttare la regola di Laplace per il calcolo effettivo del 
determinante di $A$
\begin{equation}
\label{eq:determinante}
    \det(A) = 
    \begin{cases}
        a_{11} & \text{se } n=1, \\[1em]
        \displaystyle\sum_{j=1}^{n} A_{ij}a_{ij} & \text{per } n>1.
    \end{cases}
\end{equation}

Supponendo di calcolare i determinanti tramite la \eqref{eq:determinante}, il costo computazionale della regola di Cramer \`e 
dell'ordine di $(n+1)!$ \si{\flops} (\textit{floating point operations per second}), un costo non accettabile anche per problemi di 
piccole dimensioni.\newline
Ad esempio, il \textit{supercomputer} attualmente pi\`u potente al mondo, soprannominato \enquote{El Capitan} e ospitato dal 
Lawrence Livermore National Laboratory in California (Stati Uniti)\,\cite{Thomas2024}, \`e caratterizzato da una velocit\`a pari a 
\SI{1.742e18}{\flops}, ma impiegherebbe circa \SI{2.82e40} anni\footnote{Per $n=50$, il costo computazionale \`e dell'ordine di 
$(50+1)!\simeq$\SI{1.55e66}{\flops}.\newline
Disponendo di una capacit\`a di calcolo di \SI{1.74e18}{\flops}, eseguiremmo un operazione in \SI{5.74e-19}{s}, richiedendo 
comunque un tempo di risoluzione del sistema pari a \SI{8.90e47}{s}, ovvero all'incirca \num{2.82e40} anni.}a risolvere un sistema 
lineare di $50$ equazioni con il metodo di Cramer, mentre un normale PC \`e in grado di risolvere modelli matematici con migliaia 
di vincoli in meno di un secondo, sfruttando algoritmi a elevata efficienza.

Alla luce di queste osservazioni, la necessit\`a di sviluppare metodi numerici alternativi per la risoluzione di sistemi di 
equazioni lineari \`e evidente. 
Tali metodi vengono tradizionalmente distinti in metodi
diretti se permettono la risoluzione del sistema in un numero finito di passi oppure iterativi se richiedono un numero di passi
teoricamente infinito.\newline
Preferire un metodo iterativo al posto di un metodo diretto, o viceversa, non dipende esclusivamente dall'efficienza dell'algoritmo in s\`e, ma anche dalle propriet\`a dei vettori e delle matrici coinvolte nel problema nonch\`e dalle specifiche del sistema di elaborazione relativamente alla capacit\`a di memoria e all'architettura adottata.

Nei paragrafi successivi, ci addentreremo nell'analisi dei metodi iterativi, soffermandoci in particolar modo sul metodo di Jacobi per la risoluzione di sistemi di equazioni lineari.

