%!TeX root = ../../Tesi.tex
Impiegare un metodo numerico per la risoluzione di un sistema lineare introduce necessariamente degli errori di arrotondamento, dovuti alla rappresentazione
dei numeri reali sul calcolatore con un numero finito di cifre, che fortunatamente non si ripercuotono sull'accuratezza della soluzione finale nel caso di
metodi numerici stabili, come quelli che analizzeremo.

Per una corretta quantificazione degli errori introdotti, ci avvarremo dei concetti di norma vettoriale e di norma
matriciale e del legame esistente tra quest'ultima e il raggio spettrale di una matrice.\newline
Presupponiamo, fin da subito, che questi argomenti siano familiari, cos\`i come le principali propriet\`a relative alle successioni di vettori e di matrici.
\subsection{Costruzione di un metodo iterativo}
I metodi iterativi si fondano sull'idea di calcolare una successione di vettori \\ $\{\mathbf{x}^{(k)}\in\mathbb{R}^{n}\}$, i cui elementi godano della propriet\`a di convergenza
\begin{equation}
    \label{eq:proprietaConvergenza}
    \lim_{k \to \infty} \mathbf{x}^{(k)}=\mathbf{x},
\end{equation}
dove $\mathbf{x}$ \`e la soluzione di \eqref{eq:formaMatricialeSistemiLineari}. \newline
Ovviamente, desidereremmo fermarci al minimo $m$ tale che
\begin{equation*}
    \norm{\mathbf{x}^{(m)}- \mathbf{x}} < \varepsilon,
\end{equation*}
con $\varepsilon$ una tolleranza fissata, che rappresenta il livello di accuratezza accettabile nell'approssimazione di $\mathbf{x}$, e $\norm{\cdot}$ un'opportuna norma vettoriale.

Poich\'e la soluzione esatta del sistema \eqref{eq:formaMatricialeSistemiLineari} non \`e nota a priori, introdurremo degli adeguati criteri di arresto basati
su altre grandezze nella sezione \ref{sec:criteriArresto}.

Una strategia largamente impiegata nella costruzione della successione $\{\mathbf{x}^{(k)}\}$ consiste nella decomposizione additiva della matrice dei
coefficienti $A$ in $A=P-N$, dove $P, N\in\mathbb{R}^{n \times n}$ e $P$ \`e non singolare.

In particolare, assegnato il vettore iniziale $\mathbf{x}^{(0)}$, otteniamo $\mathbf{x}^{(k)}$ per $k\ge1$ risolvendo due nuovi sistemi di $n$ equazioni in $n$
incognite
\begin{equation}
    \label{eq:metodoIterativo}
    P\mathbf{x}^{(k)}=N\mathbf{x}^{(k-1)} + \mathbf{b},\quad k\ge1
\end{equation}
Possiamo riscrivere, in maniera del tutto equivalente, la \eqref{eq:metodoIterativo} come
\begin{equation}
    \label{eq:metodoIterativoConResiduo}
    \mathbf{x}^{(k)}=\mathbf{x}^{(k-1)} + P^{-1}\mathbf{r}^{(k-1)},\quad k\ge1,
\end{equation}
avendo indicato con
\[
    \mathbf{r}^{(k-1)}=\mathbf{b}-A\mathbf{x}^{(k-1)}
\]
il vettore residuo alla $(k-1)$-esima iterazione.\newline
Usando la \eqref{eq:metodoIterativoConResiduo} per l'aggiornamento della soluzione approssimata, dobbiamo determinare il vettore residuo e risolvere un nuovo sistema lineare di
matrice $P$ a ogni iterazione.\newline
Questo procedimento risulta conveniente nell'ipotesi in cui $P$ sia invertibile con un basso costo computazionale.

Definendo
\begin{equation}
    \label{eq:errorePassoK}
    \mathbf{e}^{(k)} = \mathbf{x}^{(k)}-\mathbf{x}
\end{equation}
come l'errore al passo $k$ e osservando che, dalla decomposizione di $A$, si ricava $P\mathbf{x}= N\mathbf{x}+\mathbf{b}$, otteniamo la seguente relazione sull'errore
\begin{equation}
    \label{eq:relazioneRicorsivaErrore}
    \mathbf{e}^{(k)} = B\mathbf{e}^{(k-1)} \quad \text{con} \quad B = P^{-1}N
\end{equation}
dove $B\in\mathbb{R}^{n \times n}$ \`e chiamata matrice di iterazione associata allo \textit{splitting} $A = P - N$.
Pertanto, applicando ricorsivamente la \eqref{eq:relazioneRicorsivaErrore}, arriviamo a
\begin{equation}
    \label{eq:relazioneErroreMatriceIterazione}
    \mathbf{e}^{(k)}=B^{k}\mathbf{e}^{(0)},\quad k = 0, 1, \dots.
\end{equation}
La condizione di convergenza \eqref{eq:proprietaConvergenza} pu\`o essere riformulata in funzione dell'errore come
$\mathbf{e}^{(k)} \rightarrow \mathbf{0} \text{ per } k\rightarrow{\infty}$ soddisfatta per ogni scelta del vettore $\mathbf{x}^{(0)}$.\newline
In virt\`u della \eqref{eq:relazioneErroreMatriceIterazione}, la nuova propriet\`a di convergenza risulta verificata se e solo se ${B}^{k} \rightarrow 0 \text{ per } k\rightarrow{\infty}$.

Per quanto riguarda la convergenza di un metodo iterativo verso la soluzione esatta, esponiamo i seguenti risultati senza dimostrarli.
\begin{teorema}
    Il metodo iterativo \eqref{eq:metodoIterativo} converge alla soluzione di \eqref{eq:formaMatricialeSistemiLineari} per ogni scelta del vettore iniziale
    $\mathbf{x}^{(0)}$ se e solo se $\rho(B)<1$.
\end{teorema}
\begin{corollario}
    \label{cor:condizioneSufficienteConvergenza}
    Una condizione sufficiente per la convergenza del metodo \eqref{eq:metodoIterativo} \`e $\norm{B}<1$ per qualche norma matriciale $\norm{\cdot}$ consistente.
\end{corollario}
\begin{teorema}
    Sia $A = P - N$, con $A$ e $P$ simmetriche e definite positive. Se la matrice $2P - A$ \`e definita positiva, allora il metodo iterativo definito nella \eqref{eq:metodoIterativo} converge per ogni valore del vettore iniziale $\mathbf{x}^{(0)}$ e si ha \[\rho(B) = \norm{B}_{A} = \norm{B}_{P} < 1.\]
    Inoltre, la convergenza del metodo \`e monotona rispetto alle norme $\norm{\cdot}_{A} \ \text{e} \ \norm{\cdot}_{P}$, ovvero per ogni $k \ge 1$ valgono le seguenti relazioni
    \[
    \begin{aligned}
        \norm{\mathbf{e}^{(k+1)}}_{A} &< \norm{\mathbf{e}^{(k)}}_{A} \\
        \norm{\mathbf{e}^{(k+1)}}_{P} &< \norm{\mathbf{e}^{(k)}}_{P}.
    \end{aligned}
    \]
\end{teorema}
\subsection{Criteri di arresto per metodi iterativi}
\label{sec:criteriArresto}
Un importante argomento ancora da esaminare \`e relativo ai criteri di arresto, vale a dire le condizioni da soddisfare per decidere quando fermare l'esecuzione di un
metodo iterativo.

Un primo criterio si basa sul controllo dell'incremento: data una tolleranza $\varepsilon$ fissata, ci fermiamo al primo valore di $k$ per il quale si abbia
\begin{equation*}
    \norm{\mathbf{x}^{(k+1)} - \mathbf{x}^{(k)}}<\varepsilon,
\end{equation*}
stimando il corrispondente errore $\|\mathbf{e}^{(k+1)}\|$ all'ultima iterazione.

Sia $B$ la matrice di iterazione del metodo in esame, dalla relazione ricorsiva sull'errore $\mathbf{e}^{(k+1)}= B\mathbf{e}^{(k)}$ otteniamo
\begin{equation}
    \norm{\mathbf{e}^{(k+1)}} \le \norm{B}\,\norm{\mathbf{e}^{(k)}}.
\end{equation}
Sfruttando la disuguaglianza triangolare e il fatto che $\mathbf{e}^{(k+1)} - \mathbf{e}^{(k)} = \mathbf{x}^{(k+1)} - \mathbf{x}^{(k)}$, giungiamo a
\begin{equation*}
    \norm{\mathbf{e}^{(k+1)}} \le \norm{B}\Big(\norm{\mathbf{e}^{(k+1)}} + \norm{\mathbf{x}^{(k+1)} - \mathbf{x}^{(k)}}\Big)
\end{equation*}
e quindi (sotto l'ipotesi in cui $\|B\|<1$)
\begin{equation}
    \norm{\mathbf{x} - \mathbf{x}^{(k+1)}}\le \frac{\norm{B}}{1 - \norm{B}}\norm{\mathbf{x}^{(k+1)} - \mathbf{x}^{(k)}} \le  \frac{\norm{B}}{1 - \norm{B}} \varepsilon.
\end{equation}
Pertanto, l'errore  $\|\mathbf{e}^{(k+1)}\|$ \`e contenuto purch\`e $\|B\|\simeq 1$.

Un altro test d'arresto, pi\`u pratico dal punto di vista computazionale, si basa sul controllo del residuo normalizzato: ci fermiamo al primo valore di $k$
per il quale si ottiene $\norm{\mathbf{r}^{(k)}}/{\norm{\mathbf{r}^{(0)}}} \le \varepsilon, \text{ con } \varepsilon$ una tolleranza nota a priori.\newline
Nel caso particolare in cui $\mathbf{x}^{(0)} = \mathbf{0}$, il test richiede che
\begin{equation*}
    \frac{\|\mathbf{r}^{(k)}\|}{\|\mathbf{b}\|} \le \varepsilon.
\end{equation*}
Inoltre, possiamo quantificare l'errore relativo commesso come
\begin{equation*}
    \frac{\norm{\mathbf{x}-\mathbf{x}^{(k)}}}{\norm{\mathbf{x}}} = \frac{\norm{A^{-1}\mathbf{r}^{(k)}}}{\norm{\mathbf{x}}} = \frac{\norm{A^{-1}\mathbf{r}^{(k)}}}{\norm{\mathbf{x}}} \le K(A) \frac{\norm{\mathbf{r}^{(k)}}}{\norm{\mathbf{b}}} \le K(A)\varepsilon,
\end{equation*}
dove $K(A)=\norm{A}\,\norm{A^{-1}}$ \`e detto numero di condizionamento della matrice $A$, un indicatore circa la stabilit\`a della soluzione del sistema \eqref{eq:formaMatricialeSistemiLineari} rispetto alle perturbazioni applicate ai dati $A$ e $\mathbf{b}$.\newline
In definitiva, quest'ultimo criterio d'arresto \`e consigliato quando $K(A)\simeq 1$, ovvero quando $A$ \`e ben condizionata: piccole perturbazioni su $A \text{ e } \mathbf{b}$ implicano piccole variazioni su $\mathbf{x}$.