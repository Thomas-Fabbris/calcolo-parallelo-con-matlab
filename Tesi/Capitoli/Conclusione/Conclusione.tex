%!TeX root = ../../Tesi.tex
Nel corso di questo lavoro di tesi abbiamo fornito una panoramica sul mondo del calcolo parallelo, focalizzandoci sull'accezione di questo termine e sulle sfide portate dalla progettazione di programmi a esecuzione parallela nel capitolo \ref{cap:calcoloParalleloSfidaOpportunita}, sugli strumenti che rendono meno oneroso il processo di sviluppo di software parallelo
nel capitolo \ref{cap:unLinguaggioPerIlCalcoloParalleloMATLAB} e sull'applicazione dei concetti teorici affrontati alla risoluzione numerica di sistemi di equazioni lineari di grandi dimensioni nel capitolo \ref{cap:metodoJacobiParallelo}.

In particolare, il processo di parallelizzazione del metodo di Jacobi, terminato con la stesura dell'omonima funzione 
MATLAB, ha richiesto uno sforzo per l'applicazione dei principi di \textit{design} elencati nel paragrafo \ref{par:ingredientiMATLABParallelo}.

Pur con alcune limitazioni relative alla capacit\`a computazionale dei sistemi di elaborazione a disposizione, che non hanno 
permesso di avvicinarci alle prestazioni ipotizzate nel paragrafo \ref{par:metodoJacobi}, 
abbiamo avuto modo di apprezzare la necessit\`a di minimizzare i costi di comunicazione e di sincronizzazione tra le unit\`a di lavoro 
al fine di ottenere prestazioni soddisfacenti.%\newline
% Riguardo a questo punto, vorremmo ringraziare il \textit{Control Systems and Automation Laboratory} (CAL) dell'Universit\`a degli Studi 
% di Bergamo per aver messo a disposizione delle \textit{workstation} su cui effettuare le simulazioni, i cui risultati sono stati esposti nel paragrafo \ref{par:applicazioneMetodoJacobi}.

Un altro aspetto onnipresente durante lo svolgimento dell'elaborato \`e stato la multidisciplinariet\'a richiesta per la scrittura 
di software parallelo di qualit\'a: alla padronanza dei metodi e delle tecniche propri del calcolo parallelo, devono essere 
abbinate una profonda conoscenza del dominio applicativo in questione e una solida preparazione nell'ambito dei metodi numerici.