%!TeX root = ../../Tesi.tex
Nel corso di questo lavoro di tesi, abbiamo fornito una panoramica sul mondo del calcolo parallelo, focalizzandoci sull'accezione di questo termine e sulle sfide portate dal paradigma di programmazione da esso incarnato nel capitolo \ref{cap:calcoloParalleloSfidaOpportunita}, sugli strumenti software che possono rendere meno oneroso il processo di sviluppo di programmi a esecuzione parallela
nel capitolo \ref{cap:unLinguaggioPerIlCalcoloParalleloMATLAB} e sull'applicazione dei concetti teorici precedentemente affrontati alla risoluzione numerica di sistemi di equazioni lineari di grandi dimensioni nel capitolo \ref{cap:metodoJacobiParallelo}.

In particolare, il processo di parallelizzazione del metodo di Jacobi, culminato con la stesura dell'omonima funzione 
MATLAB riportata nell'appendice \ref{app:codiceSorgenteJacobi}, ha richiesto uno sforzo per l'applicazione dei principi di progettazione del software parallelo elencati nel paragrafo \ref{par:ingredientiMATLABParallelo}, abbattendo il labile confine tra teoria e pratica.

Pur con alcune limitazioni relative alla capacit\`a computazionale dei sistemi di elaborazione a disposizione che non hanno 
permesso di avvicinarci alle prestazioni per il metodo di Jacobi ipotizzate nel paragrafo \ref{par:metodoJacobi}, 
abbiamo potuto apprezzare la necessit\`a di minimizzare il costo di comunicazione e di sincronizzazione tra le unit\`a di lavoro 
al fine di ottenere prestazioni soddisfacenti.

Un altro aspetto onnipresente durante lo svolgimento dell'elaborato \`e stato la multidisciplinariet\'a richiesta per la scrittura 
di software parallelo di qualit\'a: alla padronanza dei metodi e delle tecniche propri del calcolo parallelo, devono essere 
abbinate una conoscenza approfondita del dominio applicativo in questione, ovvero una solida preparazione nell'ambito dei metodi numerici.